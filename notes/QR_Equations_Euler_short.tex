%% LyX 2.4.0~RC3 created this file.  For more info, see https://www.lyx.org/.
%% Do not edit unless you really know what you are doing.
\documentclass[english]{article}
\usepackage[T1]{fontenc}
\usepackage[latin9]{inputenc}
\usepackage[landscape]{geometry}
\geometry{verbose,lmargin=2cm,rmargin=2cm}
\usepackage{color}
\usepackage{amstext}
\usepackage{amssymb}

\makeatletter

%%%%%%%%%%%%%%%%%%%%%%%%%%%%%% LyX specific LaTeX commands.
%% Because html converters don't know tabularnewline
\providecommand{\tabularnewline}{\\}

\makeatother

\usepackage{babel}
\begin{document}
\title{Quantum ring growth simulation algorithm}
\maketitle

\section{Main equations}

Diffusion equations for Ga and As. $R_{d}\left(t\right)$ is the droplet
radius.

\begin{equation}
\frac{\partial C_{Ga}}{\partial t}=D_{Ga}\frac{1}{r}\frac{\partial}{\partial r}\left(r\frac{\partial C_{Ga}}{\partial r}\right)+F_{Ga}\left(r,t\right)-k_{r}C_{Ga}C_{As}
\end{equation}

\begin{equation}
\frac{\partial C_{As}}{\partial t}=D_{As}\frac{1}{r}\frac{\partial}{\partial r}\left(r\frac{\partial C_{As}}{\partial r}\right)+F_{As}-k_{r}C_{Ga}C_{As}-\frac{C_{As}}{\tau_{As}}
\end{equation}

Parameters in detail

\begin{equation}
D_{Ga}=a_{0}^{2}\nu\exp\left(-\frac{E_{Ga}}{kT}\right)
\end{equation}

\begin{equation}
D_{As}=a_{0}^{2}\nu\exp\left(-\frac{E_{As}}{kT}\right)
\end{equation}

Arsenic desorption time:

\begin{equation}
\tau_{As}=\frac{1}{\nu}\exp\left(\frac{E_{a}}{kT}\right)
\end{equation}

\begin{equation}
\nu=\frac{kT}{\pi\hbar}
\end{equation}


\subsection{Numerical values}

Let's set the distance variable in such a way that:

\[
a_{0}=1
\]

\[
\Omega_{GaAs}=1
\]

Then concentrations are per lattice cell. Droplet radius is measured
in cells. Flux is per cell per unit time.

Time units we set in such a way that:

\[
\nu\exp\left(-\frac{E_{Ga}}{kT}\right)=1
\]

\[
D_{Ga}=1
\]

\[
D_{As}=\exp\left(-\frac{E_{As}-E_{Ga}}{kT}\right)
\]

\[
\tau_{As}=\exp\left(\frac{E_{a}-E_{Ga}}{kT}\right)
\]

\begin{table}
\begin{centering}
\begin{tabular}{|c|c|c|}
\hline 
Parameter & Value & Units\tabularnewline
\hline 
\hline 
$k_{r}$ & 0.1 & cell$^{2}$/time\tabularnewline
\hline 
$a_{0}$ & 1 & cell\tabularnewline
\hline 
$h_{0}$ & 1 & cell\tabularnewline
\hline 
$\Omega_{GaAs}$ & 1 & cell$^{3}$\tabularnewline
\hline 
$\Omega_{Ga}$ & 0.1 & cell$^{3}$\tabularnewline
\hline 
$C_{0}$ & 2 & cell$^{-2}$\tabularnewline
\hline 
$E_{Ga}-E_{As}$ & 0 & eV\tabularnewline
\hline 
$E_{a}-E_{Ga}$ & 0.1 & eV\tabularnewline
\hline 
\end{tabular}
\par\end{centering}
\caption{Numerical values of the parameters}
\end{table}


\subsection{Boundary and initial conditions}

Boundary conditions for $r\in\left[0,R_{\infty}\right]$:

\[
C_{Ga}\left(R_{\infty},t\right)=0
\]

\[
C_{As}\left(R_{\infty},t\right)=F_{As}\tau_{As}
\]

Initial conditions for $r\in\left[0,R_{\infty}\right]$:

\[
C_{Ga}\left(r,0\right)=0
\]

\[
C_{As}\left(r,0\right)=0
\]


\section{Finding the Ga flux}

We require an additional concentration of Ga atoms at the droplet
boundary according to:

\begin{equation}
\tilde{C}_{Ga}\left(r,t\right)=C_{0}\exp\left(-\frac{\left(r-R_{d}\left(t\right)\right)^{2}}{w^{2}}\right)
\end{equation}

This leads to an additional flux:

\begin{equation}
F_{Ga}\left(r,t\right)=\frac{C_{0}}{\tau_{Ga}\left(t\right)}\exp\left(-\frac{\left(r-R_{d}\left(t\right)\right)^{2}}{w^{2}}\right)
\end{equation}

Which is our only source of Ga atoms! To find the parameter $\tau_{Ga}$
($w$ is a free constant value), we need to account for the loss of
Ga atoms. In a lengthy derivation, which we omit here, we obtained:

\begin{equation}
\tau_{Ga}=\frac{w^{2}}{2D_{Ga}}\left[\exp\left(-\frac{R_{d}^{2}}{w^{2}}\right)-\exp\left(-\frac{\left(R_{\infty}-R_{d}\right)^{2}}{w^{2}}\right)+\sqrt{\pi}\frac{R_{d}}{w}\left\{ \text{erf}\left(\frac{R_{\infty}-R_{d}}{w}\right)+\text{erf}\left(\frac{R_{d}}{w}\right)\right\} \right]\ln\frac{R_{\infty}}{R_{d}}
\end{equation}

If $R_{\infty}\gg R_{d}$, then this equation simplifies:

\begin{equation}
\tau_{Ga}=\frac{w^{2}}{2D_{Ga}}\left[\exp\left(-\frac{R_{d}^{2}}{w^{2}}\right)+\sqrt{\pi}\frac{R_{d}}{w}\left\{ 1+\text{erf}\left(\frac{R_{d}}{w}\right)\right\} \right]\ln\frac{R_{\infty}}{R_{d}}
\end{equation}

When $R_{d}\to0$, then naturally $\tau_{Ga}\to+\infty$ and $F_{Ga}\to0$.

So we obtain:

\begin{equation}
F_{Ga}\left(r,t\right)=\frac{2D_{Ga}C_{0}}{w^{2}}\frac{\exp\left(-\frac{\left(r-R_{d}\left(t\right)\right)^{2}}{w^{2}}\right)}{\left[\exp\left(-\frac{R_{d}^{2}}{w^{2}}\right)-\exp\left(-\frac{\left(R_{\infty}-R_{d}\right)^{2}}{w^{2}}\right)+\sqrt{\pi}\frac{R_{d}}{w}\left\{ \text{erf}\left(\frac{R_{\infty}-R_{d}}{w}\right)+\text{erf}\left(\frac{R_{d}}{w}\right)\right\} \right]\ln\frac{R_{\infty}}{R_{d}}}
\end{equation}

Note that if we define the variables:

\[
x=\frac{R_{d}}{w},\qquad p=\frac{R_{\infty}}{w}
\]

Then the function:

\[
q\left(x,p\right)=\exp\left(-x^{2}\right)-\exp\left(-\left(p-x\right)^{2}\right)+\sqrt{\pi}x\left\{ \text{erf}\left(p-x\right)+\text{erf}\left(x\right)\right\} 
\]

Can be approximated by a straight line:

\[
q\left(x,p\right)\approx a\left(p\right)x+b\left(p\right)
\]

Where:

\[
b\left(p\right)\approx\frac{0.187}{p-3.156}
\]

\[
a\left(p\right)\approx3.545
\]

So we obtain:

\begin{equation}
F_{Ga}\left(r,t\right)=\frac{2D_{Ga}C_{0}}{w^{2}}\frac{\exp\left(-\frac{\left(r-R_{d}\left(t\right)\right)^{2}}{w^{2}}\right)}{\left[3.545\frac{R_{d}}{w}+\frac{0.187w}{R_{\infty}-3.156w}\right]\ln\frac{R_{\infty}}{R_{d}}}
\end{equation}


\section{Droplet geometry}

Droplet height $H$, droplet contact angle $\theta$. Where $\theta$
is the contact angle, which could be between $30^{\circ}$ and $80^{\circ}$.

Form-factor:

\begin{equation}
B\left(\theta\right)=\frac{8-9\cos\theta+\cos3\theta}{3\sin\theta-\sin3\theta}
\end{equation}

\begin{equation}
\frac{dR_{d}}{dt}=\frac{\Omega_{Ga}}{\pi B\left(\theta\right)R_{d}^{2}}\frac{dN_{Ga}}{dt}
\end{equation}

\begin{equation}
\frac{d}{dt}\frac{R_{d}}{R_{\infty}}=-\frac{2\Omega_{Ga}D_{Ga}C_{0}}{B\left(\theta\right)R_{\infty}^{3}}\frac{R_{\infty}^{2}}{R_{d}^{2}\ln\frac{R_{\infty}}{R_{d}}}
\end{equation}

\[
\frac{R_{d}}{R_{\infty}}=\rho<1
\]

\begin{equation}
\frac{2\Omega_{Ga}D_{Ga}C_{0}}{B\left(\theta\right)R_{\infty}^{3}}=\sigma
\end{equation}

\begin{equation}
\frac{d\rho}{dt}=\frac{\sigma}{\rho^{2}\ln\rho}
\end{equation}

But this ODE can be easily solved:

\begin{equation}
\rho^{2}\ln\rho d\rho=\sigma dt
\end{equation}

\begin{equation}
\frac{\rho^{3}}{9}\left(\ln\rho^{3}-1\right)=\sigma t+C
\end{equation}

The initial state is a known value:

\begin{equation}
\rho\left(0\right)=\rho_{0}<1
\end{equation}

\begin{equation}
C=\frac{\rho_{0}^{3}}{9}\left(\ln\rho_{0}^{3}-1\right)
\end{equation}

We obtain the following equation for droplet radius:

\begin{equation}
\rho^{3}\left(\ln\rho^{3}-1\right)=\rho_{0}^{3}\left(\ln\rho_{0}^{3}-1\right)+9\sigma t
\end{equation}

Now this can be solved by Newton's method for each time.

\begin{equation}
p\left(\rho\right)=\rho^{3}\left(3\ln\rho-1\right)-\rho_{0}^{3}\left(3\ln\rho_{0}-1\right)-9\sigma t
\end{equation}

\begin{equation}
p^{\prime}\left(\rho\right)=3\rho^{2}\left(\ln\rho^{3}-1\right)-3\rho^{2}=3\rho^{2}\left(3\ln\rho-2\right)
\end{equation}

We take the first guess for $t_{i+1}$ as:

\[
\rho_{i+1}^{(0)}=\rho_{i}
\]

Then perform Newton iterations until the value stops changing:

\begin{equation}
\rho_{i+1}^{(k+1)}=\rho_{i+1}^{(k)}-\frac{\left(\rho_{i+1}^{(k)}\right)^{3}\left(3\ln\rho_{i+1}^{(k)}-1\right)-\rho_{0}^{3}\left(3\ln\rho_{0}-1\right)-9\sigma t_{i}}{3\left(\rho_{i+1}^{(k)}\right)^{2}\left(3\ln\rho_{i+1}^{(k)}-2\right)}
\end{equation}

Note that we should always have:

\begin{equation}
0<\rho<1
\end{equation}


\section{Finite difference scheme}

We introduce grid in the following form:

\[
r_{j}=j\Delta r,\quad j=0,1,\ldots,N_{r},\quad N_{r}\Delta r=R_{\infty}
\]

\[
t_{i}=i\Delta t,\quad i=0,1,\ldots,N_{t},\quad N_{t}\Delta t=T
\]

Time limit $T$ should be chosen from the condition of total droplet
depletion $R_{d}\left(T\right)\leq a$.

\subsection{Euler scheme}

\textcolor{red}{We can use Euler's method:}

\[
\frac{dx}{dt}=f\longrightarrow\frac{x_{i+1}-x_{i}}{\Delta t}=f_{i}\longrightarrow x_{i+1}=x_{i}+\Delta tf_{i}
\]

The equations become:

\[
\text{for }j=0
\]

\[
C_{Ga}^{i+1,0}=C_{Ga}^{i,0}+\Delta t\left(\frac{D_{Ga}}{\Delta r^{2}}\left[4C_{Ga}^{i,1}-4C_{Ga}^{i,0}\right]+F_{Ga}^{i,0}-k_{r}C_{Ga}^{i,0}C_{As}^{i,0}\right)
\]

\[
C_{As}^{i+1,0}=C_{As}^{i,0}+\Delta t\left(\frac{D_{As}}{\Delta r^{2}}\left[4C_{As}^{i,1}-4C_{As}^{i,0}\right]-\frac{C_{As}^{i,0}}{\tau_{As}}+F_{As}-k_{r}C_{Ga}^{i,0}C_{As}^{i,0}\right)
\]

\[
\text{for }j\in\left[1,N_{r}\right]
\]

\[
C_{Ga}^{i+1,j}=C_{Ga}^{i,j}+\Delta t\left(\frac{D_{Ga}}{\Delta r^{2}}\left[\left(1+\frac{1}{2j}\right)C_{Ga}^{i,j+1}-2C_{Ga}^{i,j}+\left(1-\frac{1}{2j}\right)C_{Ga}^{i,j-1}\right]+F_{Ga}^{i,j}-k_{r}C_{Ga}^{i,j}C_{As}^{i,j}\right)
\]

\[
C_{As}^{i+1,j}=C_{As}^{i,j}+\Delta t\left(\frac{D_{As}}{\Delta r^{2}}\left[\left(1+\frac{1}{2j}\right)C_{As}^{i,j+1}-2C_{As}^{i,j}+\left(1-\frac{1}{2j}\right)C_{As}^{i,j-1}\right]-\frac{C_{As}^{i,j}}{\tau_{As}}+F_{As}-k_{r}C_{Ga}^{i,j}C_{As}^{i,j}\right)
\]

\[
\text{for }j=N_{r}-1
\]

\[
C_{Ga}^{i,N_{r}}=0,\qquad C_{As}^{i,N_{r}}=F_{As}\tau_{As}
\]

We can divide all the equations by $C_{0}$ and further assume that
we are measuring all concentrations in units of $C_{0}$.

\[
C_{Ga}^{i+1,j}=C_{Ga}^{i,j}+\Delta t\left(\frac{D_{Ga}}{\Delta r^{2}}\left[\left(1+\frac{1}{2j}\right)C_{Ga}^{i,j+1}-2C_{Ga}^{i,j}+\left(1-\frac{1}{2j}\right)C_{Ga}^{i,j-1}\right]+\frac{F_{Ga}^{i,j}}{C_{0}}-C_{0}k_{r}C_{Ga}^{i,j}C_{As}^{i,j}\right)
\]

\[
C_{As}^{i+1,j}=C_{As}^{i,j}+\Delta t\left(\frac{D_{As}}{\Delta r^{2}}\left[\left(1+\frac{1}{2j}\right)C_{As}^{i,j+1}-2C_{As}^{i,j}+\left(1-\frac{1}{2j}\right)C_{As}^{i,j-1}\right]-\frac{C_{As}^{i,j}}{\tau_{As}}+\frac{F_{As}}{C_{0}}-C_{0}k_{r}C_{Ga}^{i,j}C_{As}^{i,j}\right)
\]

Introducing parameters:

\[
\frac{D_{Ga}\Delta t}{\Delta r^{2}}=\alpha
\]
\[
\frac{D_{As}\Delta t}{\Delta r^{2}}=\beta
\]

\[
C_{0}k_{r}\Delta t=\omega
\]

\[
\frac{\Delta r^{2}}{w^{2}}=\eta
\]

\[
\frac{\Delta tF_{Ga}^{i,j}}{C_{0}}=\frac{2\Delta tD_{Ga}}{w^{2}}\frac{\exp\left(-\frac{\left(r-R_{d}^{i}\right)^{2}}{w^{2}}\right)}{\left[3.545\frac{R_{d}^{i}}{w}+\frac{0.187w}{R_{\infty}-3.156w}\right]\ln\frac{R_{\infty}}{R_{d}^{i}}}=2\alpha\eta f_{i,j}=\upsilon f_{i,j}
\]

\[
2\alpha\eta=\upsilon
\]

\[
f_{i,j}=\frac{\exp\left(-\frac{\left(r-R_{d}^{i}\right)^{2}}{w^{2}}\right)}{\left[3.545\frac{R_{d}^{i}}{w}+\frac{0.187w}{R_{\infty}-3.156w}\right]\ln\frac{R_{\infty}}{R_{d}^{i}}}
\]

\[
\frac{\Delta tF_{As}}{C_{0}}=\kappa
\]

\[
\frac{\Delta t}{\tau_{As}}=\gamma
\]

\[
\frac{F_{As}\tau_{As}}{C_{0}}=\epsilon
\]

The equations become:

\[
\text{for }j=0
\]

\[
C_{Ga}^{i+1,0}=C_{Ga}^{i,0}+4\alpha\left[C_{Ga}^{i,1}-C_{Ga}^{i,0}\right]+2\alpha\eta f_{i,j}-\omega C_{Ga}^{i,0}C_{As}^{i,0}
\]

\[
C_{As}^{i+1,0}=C_{As}^{i,0}+4\beta\left[C_{As}^{i,1}-C_{As}^{i,0}\right]-\frac{C_{As}^{i,0}}{\tau_{As}}+g-\omega C_{Ga}^{i,0}C_{As}^{i,0}
\]

\[
\text{for }j\in\left[1,N_{r}\right]
\]

\[
C_{Ga}^{i+1,j}=C_{Ga}^{i,j}+\alpha\left[\left(1+\frac{1}{2j}\right)C_{Ga}^{i,j+1}-2C_{Ga}^{i,j}+\left(1-\frac{1}{2j}\right)C_{Ga}^{i,j-1}\right]+2\alpha\eta f_{i,j}-\omega C_{Ga}^{i,j}C_{As}^{i,j}
\]

\[
C_{As}^{i+1,j}=C_{As}^{i,j}+\beta\left[\left(1+\frac{1}{2j}\right)C_{As}^{i,j+1}-2C_{As}^{i,j}+\left(1-\frac{1}{2j}\right)C_{As}^{i,j-1}\right]-\frac{C_{As}^{i,j}}{\tau_{As}}+g-\omega C_{Ga}^{i,j}C_{As}^{i,j}
\]

\[
\text{for }j=N_{r}-1
\]

\[
C_{Ga}^{i,N_{r}}=0,\qquad C_{As}^{i,N_{r}}=\epsilon
\]


\section{Ring growth speed}

Let's derive the expression for the QR height $h\left(r,t\right)$,
which defines the QR profile at each moment in time and at each distance
from the droplet center.

Concentration of both types of atoms bound together and contributing
to the crystal growth changes according to:

\[
\frac{dC_{\text{bound}}}{dt}=k_{r}C_{Ga}C_{As}
\]

The number of GaAs vertical crystal cells at each distance increases
according to:

\[
C_{\text{bound}}=N_{\text{cells}}C_{0}
\]

\[
\frac{dN_{\text{cells}}}{dt}=\frac{k_{r}}{C_{0}}C_{Ga}C_{As}
\]

The height of the layer is connected to this number of cells:

\[
h=h_{0}N_{\text{cells}}
\]

\[
\frac{dh\left(r,t\right)}{dt}=\frac{h_{0}k_{r}}{C_{0}}C_{Ga}C_{As}
\]

Considering the scaling of the concentrations by $C_{0}$ and the
introduced parameters, we write:

\[
\frac{dh\left(r,t\right)}{dt}=h_{0}C_{0}k_{r}C_{Ga}C_{As}
\]

\[
h_{i+1,j}=h_{i,j}+\Delta th_{0}C_{0}k_{r}C_{Ga}^{i,j}C_{As}^{i,j}
\]

Finally, we obtain:

\begin{equation}
h_{i+1,j}=h_{i,j}+\omega h_{0}C_{Ga}^{i,j}C_{As}^{i,j}
\end{equation}

The initial condition is clearly:

\begin{equation}
h_{0,j}=0
\end{equation}

\begin{thebibliography}{1}
\bibitem{key-1}Z. Y. Zhou, C. X. Zheng, W. X. Tang. Origin of Quantum
Ring Formation During Droplet Epitaxy. PRL 111, 036102 (2013)

\end{thebibliography}

\end{document}

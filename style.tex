
\usepackage[russian]{babel}
\usepackage[utf8]{inputenc}
\usepackage[T2A]{fontenc}
% \usepackage{fontspec}
% \setmainfont{Times New Roman}
% \setmathfont{TeX Gyre Termes Math}
\usepackage{csquotes}

\usepackage{amsmath}
% \usepackage{unicode-math}
\renewcommand{\familydefault}{\rmdefault}
\usepackage{mathtext}
\usepackage{geometry}
\geometry{verbose,lmargin=25mm,rmargin=15mm,tmargin=15mm,bmargin=20mm}
\setcounter{secnumdepth}{3}
\setcounter{tocdepth}{3}
\usepackage{setspace}
\setstretch{1.5}

% Картинки (можно встявлять даже pdf)
\usepackage{graphicx}

% Таблицы
\usepackage{tabularx}
% \setlength{\extrarowheight}{0.3cm}
\renewcommand{\arraystretch}{1.5}
\renewcommand{\tabularxcolumn}[1]{>{\centering\arraybackslash\small}m{#1}}
\newcolumntype{B}{>{\bfseries\small}c}
\newcolumntype{R}{>{\small}c}
%% Because html converters don't know tabularnewline
\providecommand{\tabularnewline}{\\}

\makeatletter
%%%%%%%%%%%%%%%%%%%%%%%%%%%%%% Textclass specific LaTeX commands.
\numberwithin{figure}{section}
\numberwithin{table}{section}
\numberwithin{equation}{section}
\makeatother

% Абзацный отступ = 1.25см
\usepackage{indentfirst}
\setlength\parindent{12.5mm}

% Пакет для содержания
\usepackage{tocloft}

% Команда для специальных разделов (введение, обзор литературы, etc)
% Не нумеруются в содержании, по уровню вложенности: 
\newcommand{\specialsection}[1]{
    \phantomsection
    \bigskip\smallskip\hspace{-13.8mm}
    \normalfont\fontsize{18}{18}\textbf{#1}
    \par\bigskip\normalfont\normalsize
    \addcontentsline{toc}{section}{#1}
}

% Размеры заголовков разделов и подразделов
\usepackage{titlesec}
% Раздел: 18pt, добавляем слово "Глава"
\titleformat{\section}
{\fontsize{18}{18}\bfseries}{
\hspace{-1.5mm}Глава \thesection. \hskip-1em}{1em}{}
% Подраздел: 16pt
\titleformat{\subsection}
{\fontsize{16}{16}\bfseries}{\hspace{-0.2mm}\thesubsection}{1em}{}

% Содержание
% Выравнивание заголовка по центру (да, да, с отступом слева)
% т.к. окружение center и \centering не работают
\renewcommand{\cfttoctitlefont}{\hspace{0.35\textwidth} \bfseries\Large}
% \renewcommand{\cftbeforetoctitleskip}{3em}
% Слово "Глава" в содержании
\renewcommand{\cftsecpresnum}{Глава\space}
\newlength\mylength
\settowidth\mylength{\cftsecpresnum}
\addtolength\cftsecnumwidth{1.5\mylength}
% Строки с точками
\renewcommand{\cftsecleader}{\cftdotfill{\cftdotsep}}
% Точки после цифр в в содержании
\renewcommand{\cftsecaftersnum}{.}
\renewcommand{\cftsubsecaftersnum}{.}
% Подровнять subsection под точку главы
% (если глав будет больше десяти, будет чуть хуже)
\setlength{\cftsubsecindent}{2em}
% Интервал глав
\setlength{\cftbeforesecskip}{3pt}

\renewcommand{\cftsecpagefont}{\normalfont}

% Пакет, реализующий гиперссылки. Никакого расскрашивания
\usepackage[colorlinks=false,unicode=true,hidelinks]{hyperref}

\newcommand{\ITEM}{\vspace{-0.2cm}\item}
\newcommand{\MList}[1]{\par\begin{itemize}#1\end{itemize}}
\newcommand{\NList}[1]{\par\begin{enumerate}#1\end{enumerate}}

% Шрифт подписи (caption) = 12pt
% (Повезло, что small как раз равен 12pt)
\usepackage[font=small,labelfont=bf]{caption}

% Пакет, который позволяет собирать один документ TeX из нескольких
\usepackage{import}

%Библиография
\usepackage[
    backend=biber,
    %citestyle = alphabetic, 
    %bibstyle = ieee-alphabetic,  
    %sortlocale=en_US,
    sorting=none,
    backref=true,
    hyperref=true,
    style=numeric,%style=alphabetic,
    defernumbers=true,
    isbn=false,
    autolang=none,
    %eid=true,
    doi=false,
    %series=true,
    eprint=false,
    bibencoding = utf8
]{biblatex} %Imports biblatex package

\renewbibmacro{volume+number+eid}{%
    \printfield{volume}%
    \setunit{\addcomma\space}%
    \printfield{number}%
    \printfield{eid}}

\renewbibmacro{in:}{\space}

\DeclareFieldFormat[article]{volume}{{том}\space#1}
\DeclareFieldFormat[article]{number}{{номер}\space#1\addcomma}

\DefineBibliographyStrings{russian}{%
    phdthesis = {диссертация}%
}
\documentclass[14pt,oneside]{extarticle}

\usepackage[utf8]{inputenc}
\usepackage[T1,T2A]{fontenc}
% Поменял babel на polyglossia из-за проблемы с компиляцией документа на моём компьютере
\usepackage{polyglossia}
  \setdefaultlanguage{russian}
  \setotherlanguage{english}
  \newfontfamily\cyrillicfont[Script=Cyrillic]{Times New Roman}
  \newfontfamily\cyrillicfonttt[Script=Cyrillic]{Times New Roman}
% \usepackage{fontspec}
% \setmainfont{Times New Roman}
% \setmathfont{TeX Gyre Termes Math}
\usepackage{csquotes}

\usepackage{amsmath}
% \usepackage{unicode-math}
% \renewcommand{\familydefault}{\rmdefault}
\usepackage{mathtext}
\usepackage{geometry}
\geometry{verbose,lmargin=25mm,rmargin=15mm,tmargin=15mm,bmargin=20mm}
\setcounter{secnumdepth}{3}
\setcounter{tocdepth}{3}
\usepackage{setspace}
\setstretch{1.5}

% Картинки (можно встявлять даже pdf)
\usepackage{graphicx}

% Таблицы
\usepackage{tabularx}
% \setlength{\extrarowheight}{0.3cm}
\renewcommand{\arraystretch}{1.5}
\renewcommand{\tabularxcolumn}[1]{>{\centering\arraybackslash\small}m{#1}}
\newcolumntype{B}{>{\bfseries\small}c}
\newcolumntype{R}{>{\small}c}
%% Because html converters don't know tabularnewline
\providecommand{\tabularnewline}{\\}

\makeatletter
%%%%%%%%%%%%%%%%%%%%%%%%%%%%%% Textclass specific LaTeX commands.
\numberwithin{figure}{section}
\numberwithin{table}{section}
\numberwithin{equation}{section}
\makeatother

% Абзацный отступ = 1.25см
\usepackage{indentfirst}
\setlength\parindent{12.5mm}

% Пакет для содержания
\usepackage{tocloft}

% Команда для специальных разделов (введение, обзор литературы, etc)
% Не нумеруются в содержании, по уровню вложенности: 
\newcommand{\specialsection}[1]{
    \phantomsection
    \bigskip\smallskip\hspace{-13.8mm}
    \normalfont\fontsize{18}{18}\textbf{#1}
    \par\bigskip\normalfont\normalsize
    \addcontentsline{toc}{section}{#1}
}

% Размеры заголовков разделов и подразделов
\usepackage{titlesec}
% Раздел: 18pt, добавляем слово "Глава"
\titleformat{\section}
{\fontsize{18}{18}\bfseries}{
\hspace{-1.5mm}Глава \thesection. \hskip-1em}{1em}{}
% Подраздел: 16pt
\titleformat{\subsection}
{\fontsize{16}{16}\bfseries}{\hspace{-0.2mm}\thesubsection}{1em}{}

% Содержание
% Выравнивание заголовка по центру (да, да, с отступом слева)
% т.к. окружение center и \centering не работают
\renewcommand{\cfttoctitlefont}{\hspace{0.35\textwidth} \bfseries\Large}
% \renewcommand{\cftbeforetoctitleskip}{3em}
% Слово "Глава" в содержании
\renewcommand{\cftsecpresnum}{Глава\space}
\newlength\mylength
\settowidth\mylength{\cftsecpresnum}
\addtolength\cftsecnumwidth{1.5\mylength}
% Строки с точками
\renewcommand{\cftsecleader}{\cftdotfill{\cftdotsep}}
% Точки после цифр в в содержании
\renewcommand{\cftsecaftersnum}{.}
\renewcommand{\cftsubsecaftersnum}{.}
% Подровнять subsection под точку главы
% (если глав будет больше десяти, будет чуть хуже)
\setlength{\cftsubsecindent}{2em}
% Интервал глав
\setlength{\cftbeforesecskip}{3pt}

\renewcommand{\cftsecpagefont}{\normalfont}

% Пакет, реализующий гиперссылки. Никакого расскрашивания
\usepackage[colorlinks=false,unicode=true,hidelinks]{hyperref}

\newcommand{\ITEM}{\vspace{-0.2cm}\item}
\newcommand{\MList}[1]{\par\begin{itemize}#1\end{itemize}}
\newcommand{\NList}[1]{\par\begin{enumerate}#1\end{enumerate}}

% Шрифт подписи (caption) = 12pt
% (Повезло, что small как раз равен 12pt)
\usepackage[font=small,labelfont=bf]{caption}

% Пакет, который позволяет собирать один документ TeX из нескольких
\usepackage{import}

%Библиография
\usepackage[
    backend=biber,
    %citestyle = alphabetic, 
    %bibstyle = ieee-alphabetic,  
    %sortlocale=en_US,
    sorting=none,
    backref=true,
    hyperref=true,
    style=numeric,%style=alphabetic,
    defernumbers=true,
    isbn=false,
    autolang=none,
    %eid=true,
    doi=false,
    %series=true,
    eprint=false,
    bibencoding = utf8
]{biblatex} %Imports biblatex package

\renewbibmacro{volume+number+eid}{%
    \printfield{volume}%
    \setunit{\addcomma\space}%
    \printfield{number}%
    \printfield{eid}}

\renewbibmacro{in:}{\space}

\DeclareFieldFormat[article]{volume}{{том}\space#1}
\DeclareFieldFormat[article]{number}{{номер}\space#1\addcomma}

\DefineBibliographyStrings{russian}{%
    phdthesis = {диссертация}%
}

\addbibresource{literature.bib} %Import the bibliography file

% Подсветка кода (все стили в файле)
\input{code_highlight.tex}

\begin{document}

\begin{titlepage}
\begin{center}
\small{Министерство высшего образования и науки Российской федерации}

\small{Федеральное государственное автономное}

\small{образовательное учреждение высшего образования}

\small{\textbf{<<Национальный исследовательский ядерный университет <<МИФИ>>}}

\vspace{20mm}

% Название
Выпускная квалификационная работа бакалавра / магистра

\textbf{\large Название работы}

\vspace{10mm}
Направление 11.04.04 «Электроника и наноэлектроника»\\
Образовательная программа
«Название образовательной программы»

\vspace{10mm}

% Научный руководитель, рецензент
\begin{flushleft}
\textbf{Выпускник:} Фамилия И.О.

\hspace{10cm} \textit{Подпись}: \space \hrulefill

\textbf{Научный руководитель:} 

к.ф.-м.н., доцент кафедры физики конденсированных сред

ИНТЭЛ НИЯУ МИФИ, Фамилия И.О.

\hspace{10cm} \textit{Подпись}: \space \hrulefill

\textbf{И.о. заместителя заведующего кафедрой:} 

д.ф.-м.н., профессор кафедры физики конденсированных сред 

ИНТЭЛ НИЯУ МИФИ, Никитенко В.Р.

\hspace{10cm} \textit{Подпись}: \space \hrulefill \space
\end{flushleft}

\vfill

{Москва}
\par{\the\year{} г.}
\end{center}
\end{titlepage}

% Содержание
\tableofcontents
\pagebreak

% ============================================
% ВВЕДЕНИЕ
% ============================================
\specialsection{Введение}

Современные технологии наноэлектроники и квантовых устройств требуют управления морфологией наноструктур, таких как квантовые кольца, которые обладают уникальными оптическими и электронными свойствами. Капельная эпитаксия, как один из методов молекулярно-лучевой эпитаксии (МЛЭ), позволяет формировать сложные наноструктуры за счет контролируемого осаждения элементов III–V групп на полупроводниковые подложки. Этот процесс включает образование металлических капель (например, Ga или In) с последующей их кристаллизацией под воздействием потока мышьяка, что приводит к созданию квантовых точек, колец или других морфологий \cite{balakirev1}.

Несмотря на достижения в области капельной эпитаксии, остаются нерешённые вопросы, связанные с детальным пониманием механизмов формирования квантовых колец и влияния условий роста на их морфологию. В частности, необходимо исследовать, как параметры процесса, такие как температура и потоки вещества, влияют на форму и размер наноструктур. Например, исследования показали, что селективно-позиционированные наноструктуры In/GaAs(001) могут быть получены методом капельной эпитаксии на структурированных поверхностях, причём параметры роста существенно влияют на селективность и локализацию капель \cite{balakirev2}.

В данной работе рассматривается моделирование процесса капельной эпитаксии квантовых колец и исследование влияния условий роста на их форму. Для этого используются методы компьютерного моделирования, позволяющие воспроизвести процессы диффузии и кристаллизации на атомарном уровне. Подобный подход позволяет предсказать оптимальные параметры роста для получения наноструктур с заданными характеристиками и способствует более глубокому пониманию физических процессов, происходящих при капельной эпитаксии.

Квантовые кольца, полученные методом капельной эпитаксии, обладают уникальными электронными и оптическими свойствами, что делает их перспективными для использования в различных оптоэлектронных устройствах. Оптимизация технологий их создания может привести к значительному улучшению характеристик таких приборов \cite{gurioli1}.

Фотосенсоры и детекторы. Благодаря способности квантовых колец поглощать и излучать свет в широком спектральном диапазоне, они могут быть использованы для разработки высокочувствительных фотодетекторов и сенсоров \cite{ponomarenko1}. 

Лазеры и оптические антенны. Квантовые кольца представляют перспективную платформу для создания полупроводниковых лазеров нового поколения и нанофотонных устройств. Их уникальные оптические свойства, такие как возможность управления плотностью состояний и поляризационными характеристиками, позволяют улучшить пороговые параметры и эффективность лазерных систем. В частности, инновационные разработки в области оптических антенн на основе многослойных квантовых колец демонстрируют потенциал для генерации структурированного света с орбитальным угловым моментом (OAM). Например, была предложена гибридная структура из концентрических квантовых колец InGaAs/GaAs, способная формировать оптические вихри в ближнем инфракрасном диапазоне \cite{Zhang1}.

Квантовые вычисления и криптография. Благодаря дискретным энергетическим уровням и возможности контроля спиновых состояний, квантовые кольца перспективны для применения в квантовых вычислениях и криптографических системах \cite{Alfieri1}. 

Таким образом, результаты исследований в области моделирования процесса капельной эпитаксии квантовых колец и оптимизации условий их роста могут значительно повысить эффективность и функциональность различных оптоэлектронных устройств, способствуя развитию современных технологий передачи и обработки информации.

\specialsection{Цель и задачи}
\label{Tasks}

\textbf{Цель:} Основная цель дипломной работы — разработка модели капельной эпитаксии квантовых колец и анализ влияния параметров роста на их морфологию, а также написание прогаммы для визуализации и проверки этой модели. В рамках исследования были поставлены следующие задачи.

\textbf{Задачи}
\begin{enumerate}
    \item Изучить механизмы формирования колец в зависимости от температуры и интенсивности потока мышьяка и описать физическую модель данного процесса.
    \item Написать программу для получившейся физической модели.
    \item Сравнить результаты работы программы с реально выращенными квантловыми кольцами.
\end{enumerate}

Достаточно задач. Обзор литературы наверное в задачи включать не будем. Лучше написать конкретно, что мы делаем (разработка алгоритма, программная реализация, расчёт конкретных параметров при определённых условиях и т.д.)

% ============================================
% ГЛАВА 1
% ============================================
\pagebreak
\section{Обзор литературы}

\subsection{Влияние температуры роста}

Температурные условия являются одним из ключевых параметров, определяющих морфологию квантовых колец, синтезируемых методом капельной эпитаксии. Экспериментальные исследования демонстрируют, что температура роста оказывает комплексное воздействие на процессы поверхностной диффузии атомов, десорбции материала и термической стабильности структур, что в конечном итоге определяет их геометрию и воспроизводимость.

Zhou et al. (2013) исследовали образование квантовых колец GaAs при капельной эпитаксии с помощью in situ электронной микроскопии. Они показали, что кристаллизация GaAs начинается как на границе капли (формируя внутреннее кольцо), так и из-за поверхностной диффузии Ga и As (внешнее кольцо). На Рисунке 2 показан последовательный процесс образования колец при увеличении температуры.

\begin{figure}
\begin{center}
        \includegraphics[width=11cm]{images/Zhou1-Firgure1.png}
        \caption{\label{fig:Zhou1-1}
            Схематическое изображение морфологии капли и кольца показано над каждым изображением. (a) Капля Ga до воздействия потока As. (b) Под воздействием потока As за пределами капли образуется структура, похожая на юбку. (c) Капля Ga уменьшается, обнажая внутреннюю кольцевую структуру RI в исходном положении линии контакта капли. (d) Окончательное осаждение Структура с двойным кольцом из GaAs. \cite{Zhou1}.}
\end{center}
\end{figure}

\begin{figure}
\begin{center}
        \includegraphics[width=15cm]{images/Wang1-Fingure2.png}
        \caption{\label{fig:Wang1-2}
            Схема измерений спектров поглощения в поперечном поле из работы \cite{Wang1}.}
\end{center}
\end{figure}

ыыыоапапа

Wang et al. (2020) продемонстрировали, что при росте GaAs колец при различных температурах наблюдается изменение как высоты, так и плотности образующихся структур. При температуре выше 400 °C формируются не только внутренние, но и наружные кольца с разной степенью кристалличности. Это подробно проиллюстрировано на Рисунке 3 статьи, где приведены AFM-изображения колец, полученных при различных температурах.

Ростовая температура существенно влияет на морфологию квантовых колец. При низких температурах преобладает внутренняя кристаллизация, тогда как при более высоких — активизируется поверхностная диффузия, формируя внешние кольца или даже сложные структуры типа "кольцо с точкой". 

\begin{figure}
\begin{center}
    \includegraphics[width=11cm]{images/Miller2-Figure2.png}
    \caption{\label{fig:miller2-2}
        Схема измерений спектров поглощения в поперечном поле из работы \cite{miller1}.}
\end{center}
\end{figure}
    


Если рисунок взят из какой-то статьи, книги или из интернета (из интернета нежелательно), то нужно обязательно в подписи сделать ссылку на соответствующий пункт в списке литературы.
Ссылаемся на рисунок \ref{fig:Wang1-2}
Ссылаемся на рисунок \ref{fig:miller2-2}.

Рисунки в формате pdf:

\begin{figure}
    \begin{center}
        \includegraphics{images/iv-curve.pdf}
        \caption{\label{fig:iv-curve-1}
            Результаты моделирования траекторий электронов с помощью квантовых гидродинамических уравнений.}
    \end{center}
\end{figure}

\begin{figure}
    \begin{center}
        \includegraphics{images/qd_qr_potential.pdf}
        \caption{\label{fig:qr-potential-1}
            Потенциалы для моделирования КТ и КК.}
    \end{center}
\end{figure}
    
Ссылки на статьи: \cite{miller1}, \cite{miller2}, \cite{mohseni1}.

Ссылка на российскую статью: \cite{skubachevskii1}.

Ссылка на диссертацию:  \cite{pavlichenko1}

\subsection{Скорость осаждения и время роста}

Скорость подачи прекурсоров и длительность процесса роста играют решающую роль в формировании квантовых колец, определяя их геометрию, плотность и однородность. Эти параметры влияют на кинетику поверхностных процессов, включая адсорбцию атомов, их диффузию и перераспределение, что в конечном итоге определяет переход от точечных структур к кольцевым.

В исследовании Wang (2017) \cite{Wang1}, посвященном капельной эпитаксии GaAs-колец, было показано, что скорость осаждения галлия существенно влияет на морфологию структур. При низкой скорости осаждения (0.1 монослой/с) формируются кольца диаметром 80–100 нм с четкими границами, что объясняется достаточным временем для диффузии атомов и их упорядоченного распределения. Увеличение скорости до 0.5 монослой/с приводит к образованию гибридных структур: центральная область частично заполняется, формируя кольца с «размытыми» краями. При экстремально высоких скоростях (1 монослой/с) процесс переходит в кинетически ограниченный режим, что вызывает образование нерегулярных кластеров вместо колец. Авторы подчеркивают, что оптимальная скорость осаждения зависит от материала подложки: для систем на основе GaAs критическим порогом является 0.3 монослой/с, выше которого начинается деградация морфологии.

Влияние времени роста на трансформацию квантовых точек в кольца детально изучено в работе Gonzalez (2019) 
% \cite{Gonzalez1} 
для системы AlGaAs. При фиксированной температуре (400°C) увеличение времени роста с 10 до 60 секунд приводит к полному перераспределению материала из центральной точки в периферийную зону. Ключевым этапом является интервал 20–30 секунд, в течение которого происходит «растекание» капли и формирование кольцевой структуры. Однако превышение времени роста свыше 120 секунд вызывает коалесценцию соседних колец из-за увеличения радиуса диффузии атомов. Интересно, что для материалов с высокой поверхностной энергией (например, InAs) критическое время сокращается до 40–50 секунд, что требует точного контроля параметров 
%\cite{Ivanov1}
.

Противоположные результаты получены в исследованиях систем на основе нитридов. Например, в работе Tanaka (2021) 
% \cite{Tanaka1} 
для GaN-колец увеличение времени роста до 180 секунд не привело к слиянию структур благодаря высокой стабильности поверхности. Авторы связывают это с подавлением боковой диффузии за счет сильных межатомных связей в нитридных соединениях.

\subsection{Методы моделирования}

Моделирование процесса капельной эпитаксии квантовых колец является ключевым инструментом для прогнозирования морфологии структур и оптимизации условий роста. Современные подходы включают методы, основанные на различных физических принципах, от кинетических моделей до атомистических симуляций, каждый из которых обладает уникальными преимуществами и ограничениями.

Кинетические модели, такие как метод Монте-Карло (КМК), широко применяются для изучения диффузии атомов и нуклеации капель. Например, в работе Kwon (2018) 
% \cite{Kwon1} 
с помощью КМК удалось воспроизвести зависимость диаметра InGaAs-колец от температуры с погрешностью менее 10\% по сравнению с экспериментами. Авторы учитывали вероятность прыжков атомов между узлами решетки, зависимую от активационной энергии, что позволило предсказать формирование колец при температурах 350–400°C. Однако подобные модели часто игнорируют квантовые эффекты и механические напряжения, что ограничивает их применимость для сложных гетероструктур.

Молекулярная динамика (МД), опирающаяся на решение уравнений движения атомов, обеспечивает более детальное описание процессов на атомарном уровне. В исследовании Yao (2021) 
%\cite{Yao1} 
метод МД использовался для анализа роли капиллярных сил в формировании колец InGaAs. Было показано, что на ранних стадиях роста доминирует поверхностное натяжение, определяющее форму капли, тогда как на поздних стадиях ключевую роль играет адсорбция атомов на краях структуры. Несмотря на высокую точность, вычислительная сложность МД ограничивает моделирование нанометровыми системами и пикосекундными временными масштабами, что затрудняет изучение макроскопических параметров, таких как время роста.

Перспективным направлением является комбинирование методов. Например, в работе Ivanov (2022) 
%\cite{Ivanov2} 
кинетическая модель была дополнена данными МД-симуляций для учета локальных напряжений в материалах с рассогласованием решеток. Такой гибридный подход позволил предсказать образование полостей внутри InAs/GaSb-колец, что позднее было подтверждено экспериментами с помощью просвечивающей электронной микроскопии.

Современные исследования также активно внедряют машинное обучение для оптимизации параметров роста. В работе Johnson (2023) 
% \cite{Smith1} 
нейросетевая модель, обученная на данных 2000 экспериментов, предсказала оптимальные условия (температура: 380°C, скорость осаждения: 0.2 ML/с) для синтеза GaAs-колец с дисперсией диаметра менее 3\%. Однако подобные методы требуют больших объемов экспериментальных данных, что ограничивает их применение для новых материаловых систем.

\subsection{Общие закономерности}

\begin{itemize}
    \item Баланс между диффузией и десорбцией: Для каждой системы материалов существует оптимальный температурный диапазон, где скорость поверхностной диффузии достаточна для формирования колец, но десорбция не превышает критического уровня.
    \item Роль химического состава: Материалы с высокой летучестью компонентов (например, GaAs) требуют строгого контроля температуры для предотвращения десорбции, тогда как системы на основе InP демонстрируют стабильность в более широком диапазоне.
    \item Оптимизация скорости осаждения: Низкие скорости (0.1–0.3 монослой/с) обеспечивают контролируемый рост за счет преобладания диффузии над адсорбцией. Высокие скорости (>0.5 монослой/с) приводят к кинетическим ограничениям и потере структурной целостности.
    \item Критическое время роста: Для большинства арсенидных систем (InGaAs, AlGaAs) оптимальное время составляет 30–60 секунд. Превышение этого диапазона провоцирует коалесценцию, тогда как недостаточное время приводит к незавершенной трансформации точек в кольца.
    \item Упрощение термодинамики многокомпонентных сплавов: Точное описание взаимодействий между разными атомами (например, в InGaAs) остается сложной задачей.
    \item Пренебрежение трехмерными эффектами: Не учитываются такие факторы, как распределение механических напряжений в объеме структуры.
\end{itemize}

% ============================================
% ГЛАВА 2
% ============================================
\pagebreak
\section{Теория и основные уравнения}

\subsection{Раздел 1}

Ненумерованная формула:

\begin{equation}
    \begin{pmatrix} \dot{\varphi}\\ \dot{\theta} \\ \dot{\psi} \end{pmatrix}
    = \begin{pmatrix}
        \cos(\theta)\cos(\psi) & -\sin(\psi) & 0 \\
        \cos(\theta)\sin(\psi) & \cos(\psi)  & 0 \\
        -\sin(\theta)         & 0         &  1
    \end{pmatrix}^{-1}
    \begin{pmatrix} \omega_x\\ \omega_y \\ \omega_z \end{pmatrix}. \nonumber
\end{equation}


\begin{equation}
    E_{y}\left(z\geq L\right)=A_{12}e^{i\beta z}\cdot\begin{cases}
        e^{sw}\cos\left(kw\right)e^{s\left(x-a\right)}, & x<a-w\\
        \cos\left(k\left(x-a\right)\right), & a-w\leq x\le a+w\\
        e^{sw}\cos\left(kw\right)e^{-s\left(x-a\right)}, & x>a+w
        \end{cases}    
\end{equation}


\subsection{Раздел 2}

Нумерованные формулы:

\begin{equation}
\label{eq:1}
    \dot{\theta}=\frac{P-p_{1}\cos\left(\varphi_{1}-\theta\right)-p_{2}\cos\left(\varphi_{2}-\theta\right)}{\mu+\sin^{2}\left(\varphi_{1}-\theta\right)+\sin^{2}\left(\varphi_{2}-\theta\right)}
\end{equation}

\begin{equation}
    \dot{\varphi}_{1}=p_{1}-\dot{\theta}\cos(\phi_{1}-\theta)
\end{equation}

\begin{equation}
    \dot{\varphi}_{2}=p_{2}-\dot{\theta}\cos(\phi_{2}-\theta)
\end{equation}

Тест ссылки на формулу (\ref{eq:1}).

% ============================================
% ГЛАВА 3
% ============================================
\pagebreak
\section{Численные методы и алгоритмы}

\subsection{Раздел 1}

\subsection{Раздел 2}

% ============================================
% ГЛАВА 4
% ============================================
\pagebreak
\section{Программная реализация}

\begin{lstlisting}[language=rust,caption={Программная реализация метода Рунге-Кутты},label={listing-1}]
    // From the pendulum program
    fn runge_kutta(
        vars: &MyVec,
        pars: &Vec<f64>,
        rhs: &dyn Fn(&MyVec, &Vec<f64>) -> MyVec,
        dt: f64,
    ) -> MyVec {
        let rk_1 = rhs(vars, pars);
        let rk_2 = rhs(&vars.add(&rk_1.scale(dt / 2.0)), pars);
        let rk_3 = rhs(&vars.add(&rk_2.scale(dt / 2.0)), pars);
        let rk_4 = rhs(&vars.add(&rk_3.scale(dt)), pars);
    
        let vars_new = vars
            .add(&rk_1.scale(dt / 6.0))
            .add(&rk_2.scale(dt / 3.0))
            .add(&rk_3.scale(dt / 3.0))
            .add(&rk_4.scale(dt / 6.0));
        vars_new
    }
    \end{lstlisting}
    
    \begin{lstlisting}[language=C++,caption={Подпрограмма случайного блуждания на плоскости},label={listing-2}]
    std::random_device rd;
    std::mt19937 mt(rd());
    std::uniform_int_distribution<long> dist(1, 4);
    std::vector<long> xn(n0, 0);
    std::vector<long> yn(n0, 0);
    for (long jt = 0; jt < M; jt++)
    {
        for (long jn = 0; jn < n0; jn++)
        {
            switch (dist(mt))
            {
            case 1:
                xn[jn] ++;
                break;
            case 2:
                xn[jn] --;
                break;
            case 3:
                yn[jn] ++;
                break;
            case 4:
                yn[jn] --;
                break;
            }
        }
    }
    \end{lstlisting}

% ============================================
% ГЛАВА 5
% ============================================
\pagebreak
\section{Результаты и обсуждение}

Таблицы в \LaTeX ~делать очень неудобно. Лучше воспользоваться сторонним редактором таблиц, которые умеет их экспортировать в \LaTeX, сделать там всю структуру, а потом вставить готовый код, и в нём уже добавлять содержимое ячеек.

Тем не менее, простые таблицы делать можно, наподобие \ref{table-1}. Но лучше таблицами вообще не злоупотреблять, а где можно заменять их графиками и диаграммами.

\begin{center}
\begin{table}[h]
\centering{}%
\caption{Условия роста образцов с квантовыми кольцами\label{table-1}}
\begin{tabularx}{0.9\textwidth}{|B|R|R|R|R|R|X|X|}
\hline 
№ & $X_{\text{In}}$, \% & $T_1$, °C & $T_2$, °C & $P_{\text{As}_4}$, $10^{-5}$ Торр & Тип КК & \multicolumn{2}{R|}{Диаметры, нм} \tabularnewline
\hline
А1 & 0 & 220 & 220 & 1,3 & Одиночное & \multicolumn{2}{R|}{51} \tabularnewline
\hline
А2 & 0 & 280 & 280 & 0,55 & Двойное & 120 & 42 \tabularnewline
\hline
Б1 & 5 & 250 & 250 & 5,0 & Одиночное & \multicolumn{2}{R|}{75}  \tabularnewline
\hline 
Б2 & 10 & 250 & 250 & 5,0 & Одиночное & \multicolumn{2}{R|}{76}  \tabularnewline
\hline 
Б3 & 20 & 250 & 250 & 5,0 & Одиночное & \multicolumn{2}{R|}{78}  \tabularnewline
\hline 
Б4 & 20 & 200 & 200 & 5,0 & Одиночное & \multicolumn{2}{R|}{63}  \tabularnewline
\hline 
В1 & 0 & 325 & 325 & 0,2 & Одиночное & \multicolumn{2}{R|}{22}  \tabularnewline
\hline 
В2 & 0 & 325 & 220 & 0,2 & Двойное & 79 & 31 \tabularnewline
\hline 
В3 & 0 & 325 & 325 & 1,0 & Двойное & 69 & 27 \tabularnewline
\hline
\end{tabularx}
\end{table}
\end{center}

Ссылаемся на Листинг \ref{listing-1} здесь.

% ============================================
%  ВЫВОДЫ И ЗАКЛЮЧЕНИЕ
% ============================================
\pagebreak
\specialsection{Выводы}
Структура файлов, которые можно редактировать:

\begin{itemize}
    \item \verb|diploma.tex| --- содержит основной текст;
    \item \verb|titlepage.tex| --- содержит титульный лист;
    \item \verb|literature.bib| --- содержит источники для списка литературы;
    \item \verb|code_highlight.tex| --- форматирование листингов (фрагментов кода).
\end{itemize}

Файл \verb|style.tex| очень важный, его трогать и особенно удалять не надо, там задаются различные стили документа. Редактировать в случае, если знаете, что делать.

\specialsection{Заключение}

Нужны ли отдельно и выводы, и заключение --- я не знаю. Разберёмся.

Список литературы ниже оформлен не совсем по ГОСТу, но это легко исправить. Главное, что он организован, и можно ссылаться на каждый пункт по фамилии первого автора.

\textbf{Внимание!} 

Список литературы находится в отдельном файле \verb|literature.bib|, в который можно добавлять новые источники в любом порядке. Они будут сами располагаться как нужно, в порядке упоминания в тексте.

Если какой-то источник не процитирован в тексте, он в список литературы добавлен не будет.

Поэтому один и тот же файл с источниками можно использовать для нескольких документов.


\pagebreak
\printbibliography

\end{document}

\documentclass[14pt,oneside]{extarticle}

\usepackage[utf8]{inputenc}
\usepackage[T1,T2A]{fontenc}
% Поменял babel на polyglossia из-за проблемы с компиляцией документа на моём компьютере
\usepackage{polyglossia}
  \setdefaultlanguage{russian}
  \setotherlanguage{english}
  \newfontfamily\cyrillicfont[Script=Cyrillic]{Times New Roman}
  \newfontfamily\cyrillicfonttt[Script=Cyrillic]{Times New Roman}
% \usepackage{fontspec}
% \setmainfont{Times New Roman}
% \setmathfont{TeX Gyre Termes Math}
\usepackage{csquotes}

\usepackage{amsmath}
% \usepackage{unicode-math}
% \renewcommand{\familydefault}{\rmdefault}
\usepackage{mathtext}
\usepackage{geometry}
\geometry{verbose,lmargin=25mm,rmargin=15mm,tmargin=15mm,bmargin=20mm}
\setcounter{secnumdepth}{3}
\setcounter{tocdepth}{3}
\usepackage{setspace}
\setstretch{1.5}

% Картинки (можно встявлять даже pdf)
\usepackage{graphicx}

% Таблицы
\usepackage{tabularx}
% \setlength{\extrarowheight}{0.3cm}
\renewcommand{\arraystretch}{1.5}
\renewcommand{\tabularxcolumn}[1]{>{\centering\arraybackslash\small}m{#1}}
\newcolumntype{B}{>{\bfseries\small}c}
\newcolumntype{R}{>{\small}c}
%% Because html converters don't know tabularnewline
\providecommand{\tabularnewline}{\\}

\makeatletter
%%%%%%%%%%%%%%%%%%%%%%%%%%%%%% Textclass specific LaTeX commands.
\numberwithin{figure}{section}
\numberwithin{table}{section}
\numberwithin{equation}{section}
\makeatother

% Абзацный отступ = 1.25см
\usepackage{indentfirst}
\setlength\parindent{12.5mm}

% Пакет для содержания
\usepackage{tocloft}

% Команда для специальных разделов (введение, обзор литературы, etc)
% Не нумеруются в содержании, по уровню вложенности: 
\newcommand{\specialsection}[1]{
    \phantomsection
    \bigskip\smallskip\hspace{-13.8mm}
    \normalfont\fontsize{18}{18}\textbf{#1}
    \par\bigskip\normalfont\normalsize
    \addcontentsline{toc}{section}{#1}
}

% Размеры заголовков разделов и подразделов
\usepackage{titlesec}
% Раздел: 18pt, добавляем слово "Глава"
\titleformat{\section}
{\fontsize{18}{18}\bfseries}{
\hspace{-1.5mm}Глава \thesection. \hskip-1em}{1em}{}
% Подраздел: 16pt
\titleformat{\subsection}
{\fontsize{16}{16}\bfseries}{\hspace{-0.2mm}\thesubsection}{1em}{}

% Содержание
% Выравнивание заголовка по центру (да, да, с отступом слева)
% т.к. окружение center и \centering не работают
\renewcommand{\cfttoctitlefont}{\hspace{0.35\textwidth} \bfseries\Large}
% \renewcommand{\cftbeforetoctitleskip}{3em}
% Слово "Глава" в содержании
\renewcommand{\cftsecpresnum}{Глава\space}
\newlength\mylength
\settowidth\mylength{\cftsecpresnum}
\addtolength\cftsecnumwidth{1.5\mylength}
% Строки с точками
\renewcommand{\cftsecleader}{\cftdotfill{\cftdotsep}}
% Точки после цифр в в содержании
\renewcommand{\cftsecaftersnum}{.}
\renewcommand{\cftsubsecaftersnum}{.}
% Подровнять subsection под точку главы
% (если глав будет больше десяти, будет чуть хуже)
\setlength{\cftsubsecindent}{2em}
% Интервал глав
\setlength{\cftbeforesecskip}{3pt}

\renewcommand{\cftsecpagefont}{\normalfont}

% Пакет, реализующий гиперссылки. Никакого расскрашивания
\usepackage[colorlinks=false,unicode=true,hidelinks]{hyperref}

\newcommand{\ITEM}{\vspace{-0.2cm}\item}
\newcommand{\MList}[1]{\par\begin{itemize}#1\end{itemize}}
\newcommand{\NList}[1]{\par\begin{enumerate}#1\end{enumerate}}

% Шрифт подписи (caption) = 12pt
% (Повезло, что small как раз равен 12pt)
\usepackage[font=small,labelfont=bf]{caption}

% Пакет, который позволяет собирать один документ TeX из нескольких
\usepackage{import}

%Библиография
\usepackage[
    backend=biber,
    %citestyle = alphabetic, 
    %bibstyle = ieee-alphabetic,  
    %sortlocale=en_US,
    sorting=none,
    backref=true,
    hyperref=true,
    style=numeric,%style=alphabetic,
    defernumbers=true,
    isbn=false,
    autolang=none,
    %eid=true,
    doi=false,
    %series=true,
    eprint=false,
    bibencoding = utf8
]{biblatex} %Imports biblatex package

\renewbibmacro{volume+number+eid}{%
    \printfield{volume}%
    \setunit{\addcomma\space}%
    \printfield{number}%
    \printfield{eid}}

\renewbibmacro{in:}{\space}

\DeclareFieldFormat[article]{volume}{{том}\space#1}
\DeclareFieldFormat[article]{number}{{номер}\space#1\addcomma}

\DefineBibliographyStrings{russian}{%
    phdthesis = {диссертация}%
}

\addbibresource{literature.bib} %Import the bibliography file

% Подсветка кода (все стили в файле)
\input{code_highlight.tex}

\begin{document}

\begin{titlepage}
\begin{center}
\small{Министерство высшего образования и науки Российской федерации}

\small{Федеральное государственное автономное}

\small{образовательное учреждение высшего образования}

\small{\textbf{<<Национальный исследовательский ядерный университет <<МИФИ>>}}

\vspace{20mm}

% Название
Выпускная квалификационная работа бакалавра / магистра

\textbf{\large Название работы}

\vspace{10mm}
Направление 11.04.04 «Электроника и наноэлектроника»\\
Образовательная программа
«Название образовательной программы»

\vspace{10mm}

% Научный руководитель, рецензент
\begin{flushleft}
\textbf{Выпускник:} Фамилия И.О.

\hspace{10cm} \textit{Подпись}: \space \hrulefill

\textbf{Научный руководитель:} 

к.ф.-м.н., доцент кафедры физики конденсированных сред

ИНТЭЛ НИЯУ МИФИ, Фамилия И.О.

\hspace{10cm} \textit{Подпись}: \space \hrulefill

\textbf{И.о. заместителя заведующего кафедрой:} 

д.ф.-м.н., профессор кафедры физики конденсированных сред 

ИНТЭЛ НИЯУ МИФИ, Никитенко В.Р.

\hspace{10cm} \textit{Подпись}: \space \hrulefill \space
\end{flushleft}

\vfill

{Москва}
\par{\the\year{} г.}
\end{center}
\end{titlepage}

% Содержание
\tableofcontents
\pagebreak

% ============================================
% ВВЕДЕНИЕ
% ============================================
\specialsection{Введение}

Современное развитие нанотехнологий требует всё более точного контроля над структурой и формой наносистем. Особый интерес в этой области вызывают низкоразмерные квантовые структуры, такие как квантовые точки, нанонити и кольца. Среди них квантовые кольца выделяются своей замкнутой топологией и ярко выраженными квантовыми эффектами, включая эффект Ахаронова–Бома, пространственное разделение электронов и дырок, а также высокую чувствительность к внешним полям. Эти свойства делают кольцевые структуры перспективными кандидатами для использования в квантовых сенсорах, криптографии, системах хранения информации и лазерах на основе межзонных переходов~\cite{jin2010}.

Создание таких структур возможно с помощью различных подходов, включая рост по механизму Штрански–Крастанова, частичное осаждение и отжиг, но одним из наиболее гибких и управляемых методов является капельная эпитаксия. Эта технология была предложена Koguchi и соавторами в 1991 году~\cite{koguchi1991} и позволяет формировать трёхмерные наноструктуры без необходимости напряжённого слоя. Метод заключается в осаждении капель материала группы III (например, Ga) в вакуумной камере молекулярно-лучевой эпитаксии, после чего они преобразуются в кристаллические структуры за счёт воздействия атомов группы V (например, As).

Одним из ключевых преимуществ капельной эпитаксии является широкий диапазон управляемых форм, включая симметричные точки, вытянутые кольца, чашеобразные структуры и полые оболочки. Тем не менее, такие формы получаются не всегда: результаты роста чувствительны к условиям, включая температуру подложки, величину потока мышьяка, временные интервалы подачи компонентов и характеристики самой капли (объём, радиус, контактный угол). В работе~\cite{zhou2013} показано, что незначительные изменения этих параметров приводят к различным морфологиям кольца — от замкнутого симметричного обода до слабо выраженной асимметричной структуры. Аналогичную чувствительность обнаружили и Rastelli и соавт.~\cite{rastelli2004}, исследовавшие переход от квантовой точки к кольцу при частичном пассивировании и отжиге.

Формирование кольца из капли — это сложный физико-химический процесс, в котором участвуют: диффузия Ga и As по поверхности подложки, реакция между ними с образованием кристаллической решётки, десорбция мышьяка в вакуум, уменьшение радиуса капли по мере отдачи атомов Ga, а также геометрические эффекты, связанные с капельным углом и распределением давления.

Понимание этих процессов невозможно без надёжной количественной модели. Качественные гипотезы, основанные на стационарном потоке или равномерной подаче атомов, не позволяют объяснить наблюдаемое разнообразие форм. Здесь важна локальность потока атомов Ga — он сконцентрирован вблизи поверхности капли. Пространственная и временная изменчивость этого профиля напрямую влияет на геометрию кольца.

На практике такое моделирование имеет прикладное значение: оно позволяет прогнозировать форму кольца до проведения эксперимента, оптимизировать параметры роста, а также адаптировать технологию под целевые оптические и электронные свойства. Более того, численная модель может быть использована для решения обратной задачи — подбора условий роста под заданную морфологию.

Таким образом, актуальность настоящей работы определяется необходимостью построения и исследования физико-математической модели роста квантовых колец методом капельной эпитаксии, учитывающей локализованный поток атомов Ga, диффузию, реакционную кинетику, изменение формы капли и влияние внешних параметров роста.

\specialsection{Цель и задачи}

\textbf{Цель исследования} заключается в разработке и реализации численной модели капельной эпитаксии квантовых колец с учётом пространственно-временной эволюции радиуса капли и распределения концентраций атомов Ga и As, а также в анализе влияния условий роста на морфологию конечной структуры. Дополнительно целью является программная реализация модели и визуализация полученных результатов.

Для достижения указанной цели были поставлены следующие задачи:

\begin{enumerate}
  \item Исследовать физические механизмы формирования квантовых колец в рамках капельной эпитаксии, включая диффузию, десорбцию, реакционную кинетику и геометрию капли.
  \item Построить математическую модель на основе системы дифференциальных уравнений, описывающих поведение концентраций Ga и As и изменение радиуса капли во времени.
  \item Разработать численный алгоритм на основе конечно-разностной схемы (схемы Эйлера) для решения системы уравнений.
  \item Реализовать алгоритм в виде программы на языке Rust, включая:
  \begin{itemize}
    \item дискретизацию сетки по времени и пространству;
    \item вычисление профиля высоты кольца;
    \item сохранение результатов в виде файлов концентраций, высот и радиуса.
  \end{itemize}
  \item Провести серию численных экспериментов для анализа влияния параметров роста (температура, флюкс мышьяка, ширина гауссова распределения, угол смачивания) на форму кольца.
  \item Сравнить качественные характеристики полученных профилей с экспериментальными данными из научной литературы.
\end{enumerate}

% ============================================
% ГЛАВА 1
% ============================================

\pagebreak
\section{Теоретические основы капельной эпитаксии квантовых колец}

\subsection{Капельная эпитаксия: физика и стадии роста}

Капельная эпитаксия является разновидностью молекулярно-лучевой эпитаксии и используется для формирования наноразмерных структур — квантовых точек, колец, а также более сложных ансамблей. Основной особенностью метода является пространственное и временное разделение подачи атомов III и V групп. Это позволяет управлять процессом образования капель, их трансформацией и последующей кристаллизацией~\cite{gurioli2021, koguchi1991}.

Процесс капельной эпитаксии условно делится на несколько этапов:

\begin{enumerate}
    \item \textbf{Формирование капли.} На предварительно подготовленную поверхность подложки (например, GaAs) осаждаются атомы элемента III группы (чаще всего — галлия) при выключенном потоке V-группы. В отсутствие аннигиляции или немедленной реакции атомы Ga собираются в жидкие капли, распределяющиеся по поверхности в зависимости от температуры, плотности потока и состояния подложки.
    
    \item \textbf{Кристаллизация.} После формирования капли открывается поток мышьяка, и начинается стадия кристаллизации. Атомы As взаимодействуют с жидкой каплей Ga, образуя твердую фазу GaAs. На этом этапе большое значение имеет конкуренция между двумя путями роста: прямой реакцией внутри капли и кристаллизацией на поверхности в результате диффузии атомов.
    
    \item \textbf{Рост структуры.} В зависимости от соотношения этих процессов, геометрии капли и внешних условий может быть реализован рост квантовой точки, кольца или их комбинации. Так, преобладание поверхностной диффузии атомов Ga ведёт к формированию кольца, тогда как внутренняя реакция способствует образованию компактной точки.
    
    \item \textbf{Дополнительная обработка.} При необходимости структура может быть подвергнута термическому отжигу или заращиванию покровным слоем, что изменяет её морфологию, симметрию и оптические свойства. Например, частичный отжиг может привести к перераспределению материала и трансформации точки в кольцо.
\end{enumerate}

Морфология образующихся структур чувствительна к ряду технологических параметров: температуре подложки, интенсивности и длительности потока компонентов, времени экспозиции, а также к типу и ориентации подложки. Это делает капельную эпитаксию не только удобным, но и гибким инструментом для целенаправленного синтеза наноструктур с заданными свойствами~\cite{sibirmovskiy2014}.

Стадии капельной эпитаксии наглядно представлены на схематическом изображении, заимствованном из статьи~\cite{gurioli2021} (рис.~\ref{fig:gurioli1}). На нём показан процесс формирования капель, возможные пути кристаллизации и типичные формы наноструктур.

\begin{figure}
    \begin{center}
        \includegraphics[width=11cm]{images/gurioli_fig1.png}
        \caption{\label{fig:gurioli1}
            Стадии капельной эпитаксии (воспроизведено из~\cite{gurioli2021}): (a) схема последовательного осаждения атомов III и V группы, (b) AFM-изображения капель с различной плотностью, (c) два механизма роста — поглощение атомов As каплей и диффузия Ga наружу, (d) формирование квантовой точки или кольца в зависимости от доминирующего механизма.}
    \end{center}
\end{figure}

Дополнительно, на рис.~\ref{fig:zhou1} представлена последовательность реальных SEM-изображений, иллюстрирующая трансформацию капли Ga во времени под действием потока As, в том числе образование двойного кольца. Эти данные наглядно подтверждают описанный выше механизм.

\begin{figure}
    \begin{center}
        \includegraphics[width=11cm]{images/Zhou1-Firgure1.png}
        \caption{\label{fig:zhou1}
            Эволюция капли Ga и формирование квантового кольца во времени при капельной эпитаксии. Изображения получены методом MEM. Виден переход от капли к юбкообразной структуре и далее к двойному кольцу. Воспроизведено из~\cite{zhou2013}.}
    \end{center}
\end{figure}

\subsection{Квантовые кольца: геометрия, свойства, эффекты}

Квантовые кольца (КК) — это наноструктуры с тороидальной симметрией и выраженным центральным отверстием, отличающиеся от квантовых точек своей геометрией и спектром состояний. Такая топология существенно влияет на свойства локализованных в кольце электронов и дырок. В отличие от точек, где частицы локализуются в центре, в кольце они прижаты к периферии, что приводит к целому ряду квантовых эффектов.

Один из наиболее ярких эффектов — это эффект Ааронова–Бома, при котором энергетические уровни квантового кольца осциллируют в зависимости от магнитного потока, проходящего через его центр. Уравнение для энергии носителя в кольце радиуса $r$ имеет вид:

\[
E_M = \frac{\hbar^2}{2m^* r^2} \left( M + \frac{\Phi}{\Phi_0} \right)^2,
\]

где $M$ — магнитное квантовое число, $m^*$ — эффективная масса, $\Phi$ — магнитный поток, $\Phi_0 = h/e$ — квант потока.

Наличие топологического отверстия также влияет на пространственное распределение плотности состояний, запрещая плотную локализацию в центре и изменяя характер волновых функций. Это приводит к специфическим оптическим переходам: изменениям в ширине линии фотолюминесценции, поляризационной анизотропии, а также чувствительности к направлению магнитного поля.

Согласно экспериментальным данным~\cite{sibirmovskiy2018}, кольца на основе GaAs/AlGaAs демонстрируют необычную температурную зависимость фотолюминесценции: при нагревании от 20 до 70 К интенсивность ФЛ растёт, а ширина линии уменьшается, что связывается с термоактивацией носителей и неоднородной формой кольца. Кроме того, благодаря высокой симметрии кольцевой формы, КК демонстрируют слабо выраженную поляризацию испускаемого света, что делает их удобными для задач квантовой фотоники.

Энергетические и оптические свойства КК зависят от их геометрических параметров — внешнего и внутреннего радиусов, толщины, глубины. Геометрия, в свою очередь, может быть определена, например, методом атомно-силовой микроскопии. На рис.~\ref{fig:elborg2} представлен AFM-изображение одиночного квантового кольца, полученного в работе~\cite{elborg2017}.

\begin{figure}
    \begin{center}
        \includegraphics[width=11cm]{images/elborg_fig2.png}
        \caption{\label{fig:elborg2}
            AFM-изображения одиночного квантового кольца (вверху) и кольца с точкой в центре (внизу), полученных методом капельной эпитаксии. Воспроизведено из работы~\cite{elborg2017}.}
    \end{center}
\end{figure}

Квантовые кольца находят применение в квантовой криптографии, в качестве однофотонных источников, а также в схемах логических вентилей, использующих спиновые и орбитальные степени свободы. Их особые геометрические и магнитные свойства делают их также перспективными для реализации твердотельных квантовых битов.

\subsection{Влияние условий роста на морфологию квантовых колец}

Морфология квантовых колец, формируемых методом капельной эпитаксии, определяется множеством взаимосвязанных параметров. Помимо температуры подложки и потока мышьяка, большое значение имеют: объём капли, ориентация кристаллической подложки, режим подачи компонентов, параметры десорбции и свойства мокрого слоя. Управление этими факторами позволяет синтезировать кольца с заданной формой, размерами и симметрией.

Одним из наиболее исследованных параметров является температура подложки. При низких температурах ($T < 250\,^{\circ}\mathrm{C}$) ограниченная диффузия атомов Ga приводит к формированию компактных точек или толстостенных колец. При увеличении температуры ($T \approx 300\,^{\circ}\mathrm{C}$) диффузия усиливается, что способствует расширению кольца и образованию симметричной структуры. Однако при слишком высоких температурах ($T > 330\,^{\circ}\mathrm{C}$) доминирует десорбция мышьяка, что может привести к неравномерной кристаллизации и даже исчезновению кольца~\cite{sibirmovskiy2014,vasilevskiy2013}.

Интенсивность потока мышьяка (или эффективное давление As$_4$) регулирует скорость кристаллизации. Высокий поток ведёт к мгновенному насыщению капли As и образованию точек. Напротив, при умеренном или ступенчатом потоке As возможно последовательное формирование двойных или концентрических колец, как показано в работах Zhou et al.~\cite{zhou2013} и Fan и Ma~\cite{fan2023}.

\begin{figure}
    \begin{center}
        \includegraphics[width=14cm]{images/morphology_map.png}
        \caption{\label{fig:morph_map}
            Морфология квантовых колец GaAs/AlGaAs в зависимости от температуры подложки и давления мышьяка. Изображения структур заимствованы из работ~\cite{mano2005nano, koguchi2005growth, vasilevskiy2013}.}
    \end{center}
\end{figure}

Схожие результаты были получены в недавней обзорной работе Fan и Ma~\cite{fan2023}, где показано, как изменяются формы GaAs наноструктур (точки, кольца, двойные кольца, углубления) при варьировании температуры и потока As. При 150 °C формируются точки, при 200 °C — кольца, а при 300 °C наблюдаются двойные кольца и кольцевые углубления. Аналогично, при фиксированной температуре изменение потока As также приводит к трансформации форм.

\begin{figure}
    \begin{center}
        \includegraphics[width=11cm]{images/fanma_fig2e_left.png}
        \caption{\label{fig:fanma_temp}
            Изменение формы квантовых колец GaAs в зависимости от температуры. Воспроизведено из~\cite{fan2023}, рис.~2(e).}
    \end{center}
\end{figure}

\begin{figure}
    \begin{center}
        \includegraphics[width=11cm]{images/fanma_fig2e_right.png}
        \caption{\label{fig:fanma_pressure}
            Изменение формы квантовых колец GaAs в зависимости от давления мышьяка. Воспроизведено из~\cite{fan2023}, рис.~2(f).}
    \end{center}
\end{figure}

Помимо температуры подложки и потока мышьяка, на морфологию квантовых колец существенное влияние оказывают и другие параметры роста. К ним относятся:

\begin{itemize}
    \item \textbf{Объём капли Ga.} Определяет форму и ширину кольца. При малом объёме формируются узкие симметричные кольца, при увеличенном — кольца с размытыми границами или асимметрией. Существенное влияние оказывает начальный радиус и масса капли~\cite{zhou2013}.

    \item \textbf{Ориентация подложки.} На подложках GaAs(001) и GaAs(111) наблюдаются различия в симметрии кольцевых структур. Анизотропия поверхностной энергии и направленности диффузии приводит к вытягиванию кольца вдоль кристаллографических осей~\cite{elborg2017}.

    \item \textbf{Наличие смачивабщего слоя.} Если до или во время кристаллизации присутствует остаточный слой Ga или GaAs, он может «заполнять» центр кольца, делая профиль менее глубоким. В его отсутствие кольцо формируется с резкой ямой в центре~\cite{sibirmovskiy2014}.

    \item \textbf{Режим подачи мышьяка.} При непрерывной подаче формируются одиночные кольца, а при ступенчатой или импульсной — двойные и многокольцевые структуры. Режим подачи влияет на скорость кристаллизации и изменение границ капли~\cite{wang2022}.

    \item \textbf{Десорбция мышьяка.} При повышенных температурах атомы As быстро десорбируются с поверхности до завершения реакции, что может привести к неравномерной или частично разрушенной кольцевой морфологии~\cite{fan2023}.
\end{itemize}

Таким образом, морфология квантовых колец — это результат тонкого баланса между параметрами роста, диффузией, реакцией и десорбцией. Только комплексное управление этими условиями позволяет воспроизводимо получать наноструктуры с нужной формой и функциональностью.

\subsection{Современные подходы к моделированию роста квантовых колец}

Моделирование роста квантовых колец методом капельной эпитаксии остаётся активно развивающимся направлением, в котором используется широкий спектр физических и численных подходов. В данной части представлены основные классы моделей, применяемые в литературе, с анализом их возможностей и ограничений.

\subsubsection*{Модели на основе уравнений диффузии и химической реакции}
Один из наиболее распространённых подходов основан на описании эволюции поверхностных концентраций компонентов III и V групп с помощью реакционно-диффузионных уравнений. Такие модели позволяют анализировать пространственное распределение атомов и воспроизводить морфологию кольца. Классическим примером является модель, предложенная в работе Zhou et al.\cite{zhou2013}, в которой учитываются диффузия атомов Ga и As, их взаимодействие, а также влияние параметров роста. Модель хорошо воспроизводит профиль кольца, но не учитывает динамическое изменение радиуса капли, что ограничивает её применимость на длительных временах. На рис.\ref{fig:zhou_profiles} показано соответствие между профилем кольца, полученным в расчёте, и экспериментальными AFM-данными.

\begin{figure}
    \begin{center}
        \includegraphics[width=11cm]{images/zhou_profiles.png}
        \caption{\label{fig:zhou_profiles}
            Сравнение экспериментальных (слева) и рассчитанных (справа) профилей высоты кольцевых структур GaAs. Воспроизведено из~\cite{zhou2013}, рис.~4.}
    \end{center}
\end{figure}

\subsubsection*{Кинетические модели нуклеации и эволюции капель}
Другим направлением являются модели, описывающие образование и эволюцию капель на подложке. Они фокусируются на стадиях нуклеации, росте и слиянии капель, а также на влиянии температуры и плотности потока. Модель Dubrovskii et al.\cite{dubrovskii2021} представляет собой систему обыкновенных дифференциальных уравнений для плотностей мономеров, капель и критического размера. Она позволяет количественно описать начальную стадию формирования кольца, однако не рассматривает морфологию полученной структуры. На рис.\ref{fig:dubrovskii_model} показана временная эволюция параметров нуклеации согласно расчётам данной модели.

\begin{figure}
    \begin{center}
        \includegraphics[width=11cm]{images/dubrovskii_fig2.png}
        \caption{\label{fig:dubrovskii_model}
            Временная эволюция параметров нуклеации и роста капли: (a--f) — плотность мономеров, плотность капель, длина диффузии, число захвата, критический размер, объём капли. Воспроизведено из~\cite{dubrovskii2021}, рис.~2.}
    \end{center}
\end{figure}

\subsubsection*{Модели на основе метода Монте-Карло}
Для описания сложной кинетики роста на атомарном уровне применяются модели на основе кинетического Монте-Карло. Они позволяют учитывать локальные процессы осаждения, поверхностной диффузии и кристаллизации. Примером служит модель Shwartz et al.\cite{shwartz2018}, в которой смоделировано образование кольца с учётом направления поступления атомов As, перераспределения Ga и влияния геометрии капли. Такие модели хорошо воспроизводят эволюцию формы, но являются вычислительно затратными и трудны для параметрического анализа. На рис.\ref{fig:shwartz4} приведены последовательные конфигурации кольца на различных стадиях роста.

\begin{figure}
    \begin{center}
        \includegraphics[width=11cm]{images/shwartz_fig4.png}
        \caption{\label{fig:shwartz4}
            Эволюция морфологии кольца в модели Монте-Карло. Слева — поперечные сечения, справа — вид сверху. Капля Ga обозначена фиолетовым, GaAs — коричневым. Воспроизведено из~\cite{shwartz2018}, рис.~4.}
    \end{center}
\end{figure}

\subsubsection*{Сравнительный анализ подходов}
Все рассмотренные модели акцентируют внимание на различных аспектах роста кольца. Диффузионно-реакционные модели эффективны при анализе морфологии, но чувствительны к выбору граничных условий. Кинетические уравнения нуклеации хорошо описывают начальную стадию, но не дают информации о форме. Модели Монте-Карло подходят для анализа атомарных механизмов, но требуют большого объёма вычислений.

Для целей данной работы необходимо было объединить преимущества разных подходов: описать как пространственную структуру концентраций, так и изменение геометрии капли во времени. Поэтому была выбрана гибридная модель, сочетающая диффузию, химию и динамику радиуса — подробнее она представлена в следующем разделе.

\subsection{Обоснование выбора подхода к моделированию}

Анализ существующих моделей показывает, что каждая из них решает строго ограниченную задачу: описывает либо морфологию кольца при фиксированных условиях, либо начальную стадию нуклеации капли, либо пространственно-атомарную кинетику. Однако ни одна из них не объединяет в себе:

\begin{itemize}
  \item динамическое изменение радиуса капли;
  \item пространственно-временное распределение концентраций Ga и As;
  \item рост высоты кольца как результат поверхностной реакции;
  \item физически обоснованное распределение потока галлия из капли.
\end{itemize}

В связи с этим в рамках данной работы была разработана модель, сочетающая принципы реакционно-диффузионного описания, приближённое уравнение эволюции радиуса капли, численное описание роста кольца и гауссов профиль потока. Такая постановка позволяет анализировать влияние параметров роста на морфологию кольца и служит основой для построения численной схемы.

Подробное описание модели и всех входящих уравнений приводится в следующей главе.


% ============================================
% ГЛАВА 2
% ============================================
\pagebreak

\section{Физико-математическая модель}

\subsection{Основные уравнения диффузии и реакций}

Процесс формирования кольцевой структуры при капельной эпитаксии можно описывать с помощью реакционно-диффузионной модели в предположении осевой симметрии. Подобные модели применяются в ряде современных исследований~\cite{reyes2013, bietti2020}, позволяя анализировать пространственно-временное распределение компонентов и их взаимодействие на поверхности.

Пусть $c_{\text{Ga}}(r,t)$ и $c_{\text{As}}(r,t)$ — поверхностные концентрации атомов галлия и мышьяка соответственно, зависящие от радиальной координаты $r$ и времени $t$.

Для обоих компонентов учитываются диффузия, поступление вещества (для As — постоянный поток, для Ga — гауссов профиль из капли), а также реакция между ними, приводящая к образованию твёрдой фазы GaAs. Для мышьяка дополнительно учитывается десорбция с поверхности.

Такой подход описан, в частности, в работе~\cite{reyes2013_1}, где предложена численная модель, объединяющая кинетические и диффузионные аспекты. В другом исследовании~\cite{bietti2020} показано, как изменение арсенизации и температуры влияет на соотношение между объёмами GaAs, кристаллизующегося внутри капли и по её границе.

Уравнения системы имеют следующий вид:

\begin{equation}
\frac{\partial C_{Ga}}{\partial t}=D_{Ga}\frac{1}{r}\frac{\partial}{\partial r}\left(r\frac{\partial C_{Ga}}{\partial r}\right)+F_{Ga}\left(r,t\right)-k_{r}C_{Ga}C_{As}
\label{eq:ga_diff}
\end{equation}

\begin{equation}
\frac{\partial C_{As}}{\partial t}=D_{As}\frac{1}{r}\frac{\partial}{\partial r}\left(r\frac{\partial C_{As}}{\partial r}\right)+F_{As}-k_{r}C_{Ga}C_{As}-\frac{C_{As}}{\tau_{As}},
\label{eq:as_diff}
\end{equation}

где:
\begin{itemize}
  \item $D_{\text{Ga}}, D_{\text{As}}$ — коэффициенты поверхностной диффузии;
  \item $\omega$ — коэффициент реакции между атомами Ga и As;
  \item $F_{\text{As}}$ — постоянный поток мышьяка;
  \item $F_{\text{Ga}}(r,t)$ — пространственно и временно зависимый поток галлия;
  \item $\tau_{\text{As}}$ — характерное время испарения мышьяка.
\end{itemize}

\begin{equation}
D_{Ga}=a_{0}^{2}\nu\exp\left(-\frac{E_{Ga}}{kT}\right)
\end{equation}

\begin{equation}
D_{As}=a_{0}^{2}\nu\exp\left(-\frac{E_{As}}{kT}\right)
\end{equation}

Время десорбции мышьяка:

\begin{equation}
\tau_{As}=\frac{1}{\nu}\exp\left(\frac{E_{a}}{kT}\right)
\end{equation}

\begin{equation}
\nu=\frac{kT}{\pi\hbar}
\end{equation}

Модель использует приближение сплошной среды, что оправдано для масштабов порядка десятков нанометров. Она позволяет изучать влияние параметров роста на морфологию кольца в рамках эффективной численной реализации.

Модель задаётся на интервале $r \in [0, R_\infty]$ с осевой симметрией. Поверхностные концентрации атомов $C_{\text{Ga}}(r,t)$ и $C_{\text{As}}(r,t)$ описываются уравнениями~\eqref{eq:ga_diff}–\eqref{eq:as_diff}, и для их корректного численного решения необходимо задать начальные и граничные условия.

\subsubsection*{Начальные условия}

На начальный момент времени ($t = 0$): 

\begin{equation}
C_{Ga}\left(r,0\right)=0, \qquad
C_{As}\left(r,0\right)=0.
\end{equation}

\subsubsection*{Граничные условия}

На правой границе области ($r = R_\infty$) задаются:

\begin{equation}
C_{Ga}\left(R_{\infty},t\right)=0, \qquad
C_{As}\left(R_{\infty},t\right)=F_{As}\tau_{As}.
\end{equation}

\subsection{Распределение потока галлия из капли}

Одним из ключевых элементов модели капельной эпитаксии является описание пространственного распределения потока галлия, поступающего из жидкой капли на подложку. Корректное задание этой функции необходимо для воспроизведения морфологии кольца и скорости роста наноструктуры. Ниже приведено подробное аналитическое построение соответствующего выражения, опирающееся на решение уравнения диффузии с заданными граничными условиями.

Нам требуется дополнительная концентрация атомов Ga на границе капли в соответствии с:

\begin{equation}
\tilde{C}_{Ga}\left(r,t\right)=C_{0}\exp\left(-\frac{\left(r-R_{d}\left(t\right)\right)^{2}}{w^{2}}\right)
\end{equation}

Это приводит к дополнительному потоку, который является нашим единственным источником атомов Ga (рис.~\ref{fig:first}):

\begin{equation}
F_{Ga}\left(r,t\right)=\frac{C_{0}}{\tau_{Ga}\left(t\right)}\exp\left(-\frac{\left(r-R_{d}\left(t\right)\right)^{2}}{w^{2}}\right)
\end{equation}

\begin{figure}
    \begin{center}
        \includegraphics[width=11cm]{images/first.png}
        \caption{\label{fig:first}
        Визуализация потоков вещества Ga}
    \end{center}
\end{figure}

Чтобы найти параметр $\tau_{\text{Ga}}$ (где $w$ — свободное постоянное значение), нам нужно учесть потерю атомов Ga. Для этого мы рассмотрим более простую модель, в которой расчёт выполняется для $r \in [R_{d}, R_{\infty}]$ и присутствует только Ga. Это приводит к следующему уравнению:

\begin{equation}
\frac{\partial C_{Ga}}{\partial t}=D_{Ga}\frac{1}{r}\frac{\partial}{\partial r}\left(r\frac{\partial C_{Ga}}{\partial r}\right)
\end{equation}

\[
C_{\text{Ga}}\left(R_d, t\right) = C_0,
\qquad
C_{\text{Ga}}\left(R_\infty, t\right) = 0
\]

Мы можем решить эту проблему путем разделения переменных или просто рассмотреть стационарное состояние:

\[
\frac{\partial C_{Ga}}{\partial t}=0
\]

Тогда нам нужно решить:

\[
\frac{\partial}{\partial r}\left(r\frac{\partial C_{Ga}}{\partial r}\right)=0
\]

Мы получаем общее решение:

\[
C_{Ga}=A\ln\frac{r}{R_{0}}
\]

Где $A,R_{0}$ - неизвестные константы. Замена граничных условий приводит к:

\[
A\ln\frac{R_{d}}{R_{0}}=C_{0}
\qquad
A\ln\frac{R_{\infty}}{R_{0}}=0
\]

Таким образом, мы получаем:

\[
R_{0}=R_{\infty}
\]

\begin{equation}
A=\frac{C_{0}}{\ln\frac{R_{d}}{R_{\infty}}}
\end{equation}

Из чего следует:

\begin{equation}
C_{Ga}=\frac{C_{0}}{\ln\frac{R_{d}}{R_{\infty}}}\ln\frac{r}{R_{\infty}}
\end{equation}

Теперь нам нужно найти поток от границы капли:

\begin{equation}
J_{Ga}=-D_{Ga}\frac{\partial C_{Ga}}{\partial r}\left(r=R_{d}\right)
\end{equation}

\begin{equation}
\frac{\partial C_{Ga}}{\partial r}\left(r=R_{d}\right)=\frac{C_{0}}{R_{d}\ln\frac{R_{d}}{R_{\infty}}}
\end{equation}

\[
J_{Ga}=-\frac{D_{Ga}C_{0}}{R_{d}\ln\frac{R_{d}}{R_{\infty}}}
\]

Поток положительный (потому что он должен двигаться от капли в направлении r → $R_{infty}$), поэтому мы переписываем его как:

\begin{equation}
J_{Ga}=\frac{D_{Ga}C_{0}}{R_{d}\ln\frac{R_{\infty}}{R_{d}}}>0
\end{equation}

Изменение числа атомов Ga:

\begin{equation}
\frac{dN_{Ga}}{dt}=-2\pi R_{d}J_{Ga}=-2\pi\frac{D_{Ga}C_{0}}{\ln\frac{R_{\infty}}{R_{d}}}
\end{equation}

С другой стороны, в нашей первоначальной задаче мы имеем:

\begin{equation}
\frac{dN_{Ga}}{dt}=-2\pi\int_{0}^{R_{\infty}}F_{Ga}rdr
\end{equation}

\[
-\frac{dN_{Ga}}{dt}=2\pi\frac{C_{0}}{\tau_{Ga}}\int_{0}^{R_{\infty}}\exp\left(-\frac{\left(r-R_{d}\right)^{2}}{w^{2}}\right)rdr=
\]

\[
u=r-R_{d},\qquad r=u+R_{d}
\]

\begin{equation}
=2\pi\frac{C_{0}}{\tau_{Ga}}\int_{-R_{d}}^{R_{\infty}-R_{d}}\exp\left(-\frac{u^{2}}{w^{2}}\right)udu+2\pi\frac{C_{0}R_{d}}{\tau_{Ga}}\int_{-R_{d}}^{R_{\infty}-R_{d}}\exp\left(-\frac{u^{2}}{w^{2}}\right)du
\end{equation}

Благодаря симметрии:

\[
\int_{-R_{d}}^{R_{d}}\exp\left(-\frac{u^{2}}{w^{2}}\right)udu=0
\]

\[
\int_{-R_{d}}^{R_{\infty}-R_{d}}\exp\left(-\frac{u^{2}}{w^{2}}\right)udu=\int_{R_{d}}^{R_{\infty}-R_{d}}\exp\left(-\frac{u^{2}}{w^{2}}\right)udu
\]

\[
-\frac{dN_{Ga}}{dt}=\pi\frac{C_{0}w^{2}}{\tau_{Ga}}\int_{\frac{R_{d}^{2}}{w^{2}}}^{\frac{\left(R_{\infty}-R_{d}\right)^{2}}{w^{2}}}\exp\left(-v\right)dv+2\pi\frac{C_{0}wR_{d}}{\tau_{Ga}}\int_{-\frac{R_{d}}{w}}^{\frac{R_{\infty}-R_{d}}{w}}\exp\left(-v^{2}\right)dv
\]

\begin{equation}
    \begin{split}
    -\frac{dN_{Ga}}{dt} &= \pi \frac{C_{0} w^{2}}{\tau_{Ga}} \left[
    \exp\left(-\frac{R_{d}^{2}}{w^{2}}\right)
    - \exp\left(-\frac{(R_{\infty} - R_{d})^{2}}{w^{2}}\right) \right. \\
    &\quad + \left. \sqrt{\pi} \frac{R_{d}}{w}
    \left\{ \text{erf}\left(\frac{R_{\infty} - R_{d}}{w}\right)
    + \text{erf}\left(\frac{R_{d}}{w}\right) \right\}
    \right]
    \end{split}
\end{equation}
    

Сравнение двух потоков дает нам следующее уравнение:

\[
    \begin{split}
    \frac{D_{Ga}}{\ln\frac{R_{\infty}}{R_{d}}} &= \frac{1}{2} \cdot \frac{w^{2}}{\tau_{Ga}} \left[
    \exp\left(-\frac{R_{d}^{2}}{w^{2}}\right)
    - \exp\left(-\frac{(R_{\infty} - R_{d})^{2}}{w^{2}}\right) \right. \\
    &\quad + \left. \sqrt{\pi} \cdot \frac{R_{d}}{w}
    \left\{ \text{erf}\left(\frac{R_{\infty} - R_{d}}{w}\right)
    + \text{erf}\left(\frac{R_{d}}{w}\right) \right\}
    \right]
    \end{split}
\]
    

\begin{equation}
    \begin{split}
    \tau_{Ga} = \frac{w^{2}}{2 D_{Ga}} \left[
    \exp\left(-\frac{R_{d}^{2}}{w^{2}}\right)
    - \exp\left(-\frac{(R_{\infty} - R_{d})^{2}}{w^{2}}\right) \right. \\
    \left. + \sqrt{\pi} \cdot \frac{R_{d}}{w}
    \left\{ \text{erf}\left(\frac{R_{\infty} - R_{d}}{w}\right)
    + \text{erf}\left(\frac{R_{d}}{w}\right) \right\}
    \right] \cdot \ln\left(\frac{R_{\infty}}{R_{d}}\right)
    \end{split}
\end{equation}

\begin{equation}
\tau_{Ga}=\frac{w^{2}}{2D_{Ga}}\left[\exp\left(-\frac{R_{d}^{2}}{w^{2}}\right)+\sqrt{\pi}\frac{R_{d}}{w}\left\{ 1+\text{erf}\left(\frac{R_{d}}{w}\right)\right\} \right]\ln\frac{R_{\infty}}{R_{d}}
\end{equation}

При $R_{\infty}\gg R_{d}$, то это уравнение упрощает:

\begin{equation}
\tau_{Ga}=\frac{w^{2}}{2D_{Ga}}\left[\exp\left(-\frac{R_{d}^{2}}{w^{2}}\right)+\sqrt{\pi}\frac{R_{d}}{w}\left\{ 1+\text{erf}\left(\frac{R_{d}}{w}\right)\right\} \right]\ln\frac{R_{\infty}}{R_{d}}
\end{equation}

Когда $R_{d}\to0$, то, естественно, $\tau_{Ga}\to+\infty$ и $F_{Ga}\to0$:

\begin{equation}
F_{Ga}\left(r,t\right)=\frac{2D_{Ga}C_{0}}{w^{2}}\frac{\exp\left(-\frac{\left(r-R_{d}\left(t\right)\right)^{2}}{w^{2}}\right)}{\left[\exp\left(-\frac{R_{d}^{2}}{w^{2}}\right)-\exp\left(-\frac{\left(R_{\infty}-R_{d}\right)^{2}}{w^{2}}\right)+\sqrt{\pi}\frac{R_{d}}{w}\left\{ \text{erf}\left(\frac{R_{\infty}-R_{d}}{w}\right)+\text{erf}\left(\frac{R_{d}}{w}\right)\right\} \right]\ln\frac{R_{\infty}}{R_{d}}}
\end{equation}

Обратите внимание, что если мы переобозначим переменные:

\[
x=\frac{R_{d}}{w},\qquad p=\frac{R_{\infty}}{w}
\]

Тогда функция:

\[
q\left(x,p\right)=\exp\left(-x^{2}\right)-\exp\left(-\left(p-x\right)^{2}\right)+\sqrt{\pi}x\left\{ \text{erf}\left(p-x\right)+\text{erf}\left(x\right)\right\} 
\]

Может быть аппроксимирована к прямой линии:

\[
q\left(x,p\right)\approx a\left(p\right)x+b\left(p\right)
\]

Где:

\[
b\left(p\right)\approx\frac{0.187}{p-3.156}
\]

\[
a\left(p\right)\approx3.545
\]

Таким образом, мы получаем:

\begin{equation}
F_{Ga}\left(r,t\right)=\frac{2D_{Ga}C_{0}}{w^{2}}\frac{\exp\left(-\frac{\left(r-R_{d}\left(t\right)\right)^{2}}{w^{2}}\right)}{\left[3.545\frac{R_{d}}{w}+\frac{0.187w}{R_{\infty}-3.156w}\right]\ln\frac{R_{\infty}}{R_{d}}}
\end{equation}

\subsection{Геометрия капли и коэффициент $B(\theta)$}

Для корректного описания изменения объёма жидкой капли галлия при капельной эпитаксии используется приближение сферического сегмента с фиксированным углом смачивания $\theta$. Такая капля рассматривается как часть сферы радиуса $R$, с высотой $h$ и радиусом основания $R_d$ (см. рис.~\ref{fig:drop_geom}).

\begin{figure}
    \begin{center}
    \includegraphics[width=11cm]{images/contact_angle_schematic.png}
    \caption{\label{fig:drop_geom}
    Геометрия сферической капли: $r(R_d)$ — радиус контактной области, $h$ — высота, $\theta$ — угол смачивания. Воспроизведено из работы~\cite{aboubakri2021}.}
    \end{center}
\end{figure}

Классическая формула объёма сферического сегмента имеет вид:
\begin{equation}
V = \frac{1}{3} \pi h^2 (3R - h),
\end{equation}

\[
sin\theta = \frac{R_d}{R}
\qquad
ctg\theta = \frac{R-h}{R_d}
\]

откуда $h = R(1 - \cos\theta)$.

Подставим:
\[
V = \frac{1}{3} \pi R^3 (1 - \cos\theta)^2 (2 + \cos\theta).
\]

Поскольку $R = R_d / \sin\theta$, то $R^3 = R_d^3 / \sin^3\theta$. Подставим это в выражение для $V$:
\[
V = \frac{1}{3} \pi \cdot \frac{R_d^3}{\sin^3 \theta} \cdot (1 - \cos\theta)^2 (2 + \cos\theta).
\]

Теперь введём коэффициент $B(\theta)$:
\begin{equation}
V = \frac{B(\theta)\pi R_d^3}{3}.
\end{equation}
Сравнивая, получаем:
\[
B(\theta) = \frac{\left(1-\cos\theta\right)^{2}\left(2+\cos\theta\right)}{\sin^{3}\theta}.
\]

Это выражение можно упростить до более компактной формы с использованием тригонометрических тождеств:
\[
\frac{\left(1-\cos\theta\right)^{2}\left(2+\cos\theta\right)}{\sin^{3}\theta}=\frac{\left(1-2\cos\theta+\cos^{2}\theta\right)\left(2+\cos\theta\right)}{\sin^{3}\theta}=
\]

\[
=\frac{2-4\cos\theta+2\cos^{2}\theta+\cos\theta-2\cos^{2}\theta+\cos^{3}\theta}{\sin^{3}\theta}=
\]

\[
=\frac{2-3\cos\theta+\frac{3}{4}\cos\theta+\frac{1}{4}\cos3\theta}{\frac{3}{4}\sin\theta-\frac{1}{4}\sin3\theta}=\frac{8-9\cos\theta+\cos3\theta}{3\sin\theta-\sin3\theta}
\]

\begin{equation}
B(\theta) = \frac{8-9\cos\theta+\cos3\theta}{3\sin\theta-\sin3\theta}
.
\label{eq:Btheta_final}
\end{equation}

Именно это выражение реализовано в численном коде и используется во всех вычислениях, связанных с объёмом капли.

\subsection{Рост кольца}

Рост квантового кольца происходит в результате поверхностной реакции между атомами галлия и мышьяка, приводящей к образованию кристаллической фазы GaAs. Локальное увеличение высоты кольца напрямую связано с числом образованных ячеек кристалла в данной точке.

Концентрация обоих типов атомов, связанных друг с другом и участвующих в росте кристалла, изменяется в соответствии с:

\[
\frac{dC_{\text{bound}}}{dt} = k_r \, C_{\text{Ga}}(r,t) \cdot C_{\text{As}}(r,t)
\]
где:
\begin{itemize}
  \item $C_{\text{bound}}$ — концентрация результата реакции (GaAs);
  \item $k_r$ — константа скорости реакции;
  \item $C_{\text{Ga}}, C_{\text{As}}$ — локальные концентрации атомов;
\end{itemize}

Количество вертикальных кристаллических ячеек GaAs на каждом расстоянии увеличивается в зависимости от:

\[
C_{\text{bound}} = N_{\text{cells}} \cdot C_0 \quad \Rightarrow \quad
\frac{dN_{\text{cells}}}{dt} = \frac{k_r}{C_0} \, C_{\text{Ga}} \cdot C_{\text{As}}.
\]
где:
\begin{itemize}
  \item $C_0 = \frac{1}{a_0^2}$ , $a_0$ - постоянная решётки GaAs;
  \item $N_{\text{cells}}$ — количество слоёв ячеек GaAs, выросших в данной точке (вдоль вертикали);
\end{itemize}

Высота слоя связана с этим количеством ячеек:

\[
    h=h_{0}N_{\text{cells}}
\]

\[
\frac{dh\left(r,t\right)}{dt}=\frac{h_{0}k_{r}}{C_{0}}C_{Ga}C_{As}
\]
где $h_0=a_0$ — высота одного элементарного слоя (порядка постоянной решётки);

Интегрируя это уравнение по времени, можно получить полный профиль кольца. Поскольку значения $C_{\text{Ga}}$ и $C_{\text{As}}$ варьируются по координате $r$ и времени $t$, итоговая форма кольца чувствительна к условиям роста, включая поток, температуру и геометрию капли.

\subsection{Уравнение изменения радиуса капли \texorpdfstring{$R_d(t)$}{Rd(t)}}

Изменение радиуса жидкой капли галлия во времени связано с постепенной кристаллизацией GaAs в результате взаимодействия атомов Ga и As на поверхности.
\begin{equation}
\frac{\partial C_{\text{bound}}}{\partial t}=k_{r}C_{Ga}C_{As}
\end{equation}

Это должен быть единственный способ, которым атомы Ga могут исчезнуть (за исключением тех, которые выходят за пределы границы, но мы пока отбросим их). 

Скорость изменения объёма капли пропорциональна количеству атомов, покидающих её в результате химической реакции. Если \( C_{\text{Ga}}(r,t) \) и \( C_{\text{As}}(r,t) \) — локальные поверхностные концентрации атомов галлия и мышьяка, то число атомов, вовлечённых в рост GaAs на малом интервале времени \( dt \), определяется как:

\begin{equation}
\frac{dN_{Ga}}{dt}=-2\pi k_{r}C_{0}^{2}\int_{0}^{R_{\infty}}rC_{Ga}C_{As}dr
\end{equation}

Рассмотрим сферическую каплю объёмом
\begin{equation}
V = \frac{B(\theta)\pi R_d^3}{3},
\end{equation}

Дифференцируя объём по времени, получаем:
\begin{equation}
\frac{dV}{dt} = {\pi R_d^2B(\theta)} \cdot \frac{dR_d}{dt}.
\label{eq:dVdt_from_R}
\end{equation}

С другой стороны, объём капли можно выразить через количество атомов галлия:
\begin{equation}
V = \Omega_{\text{Ga}} \cdot N_{\text{Ga}} \quad \Rightarrow \quad \frac{dV}{dt} = \Omega_{\text{Ga}} \cdot \frac{dN_{\text{Ga}}}{dt},
\label{eq:dVdt_from_N}
\end{equation}
где $\Omega_{\text{Ga}} = a_0^3$ — объём одного атома галлия.

Подставляя \eqref{eq:dVdt_from_N} в \eqref{eq:dVdt_from_R}, получаем:
\begin{equation}
\frac{dR_d}{dt} = \frac{\Omega_{\text{Ga}} }{\pi R_d^2 B(\theta)} \cdot \frac{dN_{\text{Ga}}}{dt}.
\end{equation}

Мы получаем:

\begin{equation}
\frac{dR_{d}}{dt}=-\frac{2\Omega_{Ga}k_{r}C_{0}^{2}}{B\left(\theta\right)R_{d}^{2}}\int_{0}^{R_{\infty}}rC_{Ga}C_{As}dr
\label{eq:dRdt_integral}
\end{equation}
где:
\begin{itemize}
  \item \( \Omega_{\text{Ga}} = a_0^3 \) — объём одного атома галлия;
  \item \( R_\infty \) — граница области расчёта.
\end{itemize}

При численной реализации интеграл в уравнении \eqref{eq:dRdt_integral} заменяется на дискретную сумму по координатной сетке, как будет показано в главе 3. Тем не менее, приведённая здесь непрерывная форма демонстрирует физическую суть: скорость уменьшения радиуса капли определяется пространственно распределённой реакцией атомов Ga и As на поверхности.

% ============================================
% ГЛАВА 3
% ============================================

\pagebreak
\section{Численные методы и реализация}

\subsection{Реализация схемы Эйлера}

В настоящей работе для численного решения уравнений диффузии и реакции была выбрана явная схема Эйлера. Её преимуществами являются простота реализации и прозрачность численных операций. При соблюдении условий устойчивости (см. уравнение 3.2) схема обеспечивает адекватное поведение решения и позволяет эффективно исследовать динамику на относительно коротких временах~\cite{usheva2014}.

\subsubsection{Пространственно-временная сетка}

Для численного решения модели расчётная область по радиальной координате \( r \in [0, R_\infty] \) разбивается на равномерную сетку с шагом \( \Delta r \). Точки сетки обозначаются как \( r_j = j \cdot \Delta r \), где \( j = 0, 1, ..., N_r \), и \( N_r \) — число интервалов. Соответственно, \( R_\infty = N_r \cdot \Delta r \).

Временная эволюция описывается дискретными шагами по времени с шагом \( \Delta t \). Точки сетки обозначаются как \( t_{i}=i\cdot\Delta t \), где \( i=0,1,\ldots,N_{t} \), и \( N_t \) — число интервалов. 

\paragraph{Условие устойчивости явной схемы.}

При использовании явной схемы Эйлера для решения уравнений диффузии необходимо обеспечить выполнение условия устойчивости, чтобы избежать накопления ошибок и коллапса решения.

Для одномерного уравнения диффузии в декартовых координатах:
\[
\frac{\partial c}{\partial t} = D \cdot \frac{\partial^2 c}{\partial x^2},
\]
явная схема Эйлера имеет вид:
\[
\frac{c_i^{j+1} - c_i^j}{\Delta t} = D \cdot \frac{c_{i+1}^j - 2c_i^j + c_{i-1}^j}{\Delta x^2}.
\]

Для устойчивости такой схемы необходимо\cite{leveque2007}, чтобы:
\begin{equation}
\Delta t < \frac{1}{2} \cdot \frac{\Delta x^2}{D}.
\label{eq:cfl_classic}
\end{equation}

В случае полярных координат с радиальной симметрией:
\[
\frac{\partial c}{\partial t} = D \left(
\frac{\partial^2 c}{\partial r^2} + \frac{1}{r} \cdot \frac{\partial c}{\partial r}
\right),
\]
вклад первой производной по \( r \) не оказывает доминирующего влияния на устойчивость. Поэтому основное ограничение на шаг времени остаётся тем же, что и в декартовой схеме.

С учётом того, что в уравнении модели участвуют два компонента (Ga и As) с различными коэффициентами диффузии, шаг времени должен быть подобран так, чтобы обеспечивать устойчивость для обоих уравнений. Поэтому берётся наиболее жёсткое ограничение:

\begin{equation}
\Delta t < \frac{1}{2} \cdot \frac{\Delta r^2}{\max(D_{\text{Ga}}, D_{\text{As}})}
\label{eq:cfl_final}
\end{equation}

Таким образом, пространственно-временная сетка и дискретизация позволяют перейти от непрерывных уравнений физической модели к вычислимым численным алгоритмам, которые будут реализованы в последующих разделах.

\subsubsection{Основные уравнения диффузии и реакций и поток галлия}

Аппроксимация уравнений реакции-диффузии зависит от положения узла сетки \( j \), и все выражения можно разбить на три случая:

\paragraph{1. Узел \( j = 0 \) (центр симметрии)}

В этом случае диффузионный член учитывает симметрию с удвоенным весом:

\[
C_{\text{Ga}}^{i+1,0} = C_{\text{Ga}}^{i,0}
+ \Delta t \left(
\frac{D_{\text{Ga}}}{\Delta r^2} [4 C_{\text{Ga}}^{i,1} - 4 C_{\text{Ga}}^{i,0}]
+ F_{\text{Ga}}^{i,0}
- k_r C_{\text{Ga}}^{i,0} C_{\text{As}}^{i,0}
\right),
\]

\[
C_{\text{As}}^{i+1,0} = C_{\text{As}}^{i,0}
+ \Delta t \left(
\frac{D_{\text{As}}}{\Delta r^2} [4 C_{\text{As}}^{i,1} - 4 C_{\text{As}}^{i,0}]
- \frac{C_{\text{As}}^{i,0}}{\tau_{\text{As}}}
+ F_{\text{As}} - k_r C_{\text{Ga}}^{i,0} C_{\text{As}}^{i,0}
\right).
\]

\paragraph{2. Внутренние узлы \( j \in [1, N_r-1] \)}

Для всех остальных внутренних узлов используется стандартная центральная аппроксимация с учётом полярных координат:

\[
C_{\text{Ga}}^{i+1,j} = C_{\text{Ga}}^{i,j}
+ \Delta t \left(
\frac{D_{\text{Ga}}}{\Delta r^2}
\left[ \left(1+\frac{1}{2j}\right) C_{\text{Ga}}^{i,j+1}
- 2 C_{\text{Ga}}^{i,j}
+ \left(1-\frac{1}{2j}\right) C_{\text{Ga}}^{i,j-1} \right]
+ F_{\text{Ga}}^{i,j}
- k_r C_{\text{Ga}}^{i,j} C_{\text{As}}^{i,j}
\right),
\]

\[
C_{\text{As}}^{i+1,j} = C_{\text{As}}^{i,j}
+ \Delta t \left(
\frac{D_{\text{As}}}{\Delta r^2}
\left[ \left(1+\frac{1}{2j}\right) C_{\text{As}}^{i,j+1}
- 2 C_{\text{As}}^{i,j}
+ \left(1-\frac{1}{2j}\right) C_{\text{As}}^{i,j-1} \right]
- \frac{C_{\text{As}}^{i,j}}{\tau_{\text{As}}}
+ F_{\text{As}}
- k_r C_{\text{Ga}}^{i,j} C_{\text{As}}^{i,j}
\right).
\]

\paragraph{3. Граничные условия на \( j = N_r \)}

На правой границе области применяются условия:
\[
C_{\text{Ga}}^{i,N_r} = 0, \qquad C_{\text{As}}^{i,N_r} = F_{\text{As}} \cdot \tau_{\text{As}}.
\]

\paragraph{4. Переход к безразмерной форме}

Для упрощения записи вводятся нормированные концентрации \( c = C / C_0 \) и безразмерные параметры:

\[
\alpha = \frac{D_{\text{Ga}} \Delta t}{\Delta r^2}, \quad
\beta = \frac{D_{\text{As}} \Delta t}{\Delta r^2}, \quad
\omega = C_0 k_r \Delta t, \quad
\gamma = \frac{\Delta t}{\tau_{\text{As}}}, \quad
\kappa = \frac{\Delta t F_{\text{As}}}{C_0}, \quad
\epsilon = \frac{F_{\text{As}} \tau_{\text{As}}}{C_0}.
\]

Также вводятся параметры, связанные с потоком галлия:

\[
\eta = \frac{\Delta r^2}{w^2}, \quad
f_{i,j} = \frac{\exp\left( -\frac{(r_j - R_d^i)^2}{w^2} \right)}
{\left[3.545 \frac{R_d^i}{w} + \frac{0.187 w}{R_\infty - 3.156 w}\right] \ln\left( \frac{R_\infty}{R_d^i} \right)}, \quad
\upsilon = 2 \alpha \eta.
\]

Итоговые выражения для нормированных концентраций для $j=0$:

\[
C_{\text{Ga}}^{i+1,0} = C_{\text{Ga}}^{i,0}
+ 4 \alpha (C_{\text{Ga}}^{i,1} - C_{\text{Ga}}^{i,0})
+ \upsilon f_{i,0}
- \omega C_{\text{Ga}}^{i,0} C_{\text{As}}^{i,0},
\]

\[
C_{\text{As}}^{i+1,0} = C_{\text{As}}^{i,0}
+ 4 \beta (C_{\text{As}}^{i,1} - C_{\text{As}}^{i,0})
- \gamma C_{\text{As}}^{i,0}
+ \kappa
- \omega C_{\text{Ga}}^{i,0} C_{\text{As}}^{i,0},
\]

для $j\in\left[1,N_{r-1}\right]$:

\[
C_{\text{Ga}}^{i+1,j} = C_{\text{Ga}}^{i,j}
+ \alpha \left( \left(1+\frac{1}{2j}\right) C_{\text{Ga}}^{i,j+1}
- 2 C_{\text{Ga}}^{i,j}
+ \left(1-\frac{1}{2j}\right) C_{\text{Ga}}^{i,j-1} \right)
+ \upsilon f_{i,j}
- \omega C_{\text{Ga}}^{i,j} C_{\text{As}}^{i,j},
\]

\[
C_{\text{As}}^{i+1,j} = C_{\text{As}}^{i,j}
+ \beta \left( \left(1+\frac{1}{2j}\right) C_{\text{As}}^{i,j+1}
- 2 C_{\text{As}}^{i,j}
+ \left(1-\frac{1}{2j}\right) C_{\text{As}}^{i,j-1} \right)
- \gamma C_{\text{As}}^{i,j}
+ \kappa
- \omega C_{\text{Ga}}^{i,j} C_{\text{As}}^{i,j}.
\]

для $j=N_{r}$:

\[
C_{\text{Ga}}^{i,N_r} = 0, \qquad C_{\text{As}}^{i,N_r} = \epsilon.
\]

\subsubsection{Рост кольца}

С учётом масштабирования по \( C_0 \), рост высоты кольца можно записать как:

\begin{equation}
\frac{d h(r,t)}{d t} = h_0 \cdot C_0 \cdot k_r \cdot C_{\text{Ga}}(r,t) \cdot C_{\text{As}}(r,t),
\end{equation}

Для численного расчёта это уравнение дискретизуется по времени с использованием явной схемы Эйлера:

\begin{equation}
h_{i+1,j} = h_{i,j} + \Delta t \cdot h_0 \cdot C_0 \cdot k_r \cdot C_{\text{Ga}}^{i,j} \cdot C_{\text{As}}^{i,j}.
\end{equation}

С учётом ранее введённого безразмерного параметра \( \omega = C_0 k_r \Delta t \), уравнение принимает окончательный вид:

\begin{equation}
h_{i+1,j} = h_{i,j} + \omega h_0 C_{\text{Ga}}^{i,j} C_{\text{As}}^{i,j}.
\end{equation}

Начальное условие для высоты кольца задаётся как:

\begin{equation}
h_{0,j} = 0.
\end{equation}

\subsubsection{Изменение радиуса капли}

Основываясь на объёме сферического сегмента и скорости расхода атомов, можно записать уравнение \eqref{eq:dRdt_integral} в виде:

\begin{equation}
R_d^{i+1} = R_d^i - \frac{2 \Omega_{\text{Ga}} k_r C_0^2 \Delta t \Delta r^2}{B(\theta) (R_d^i)^2} \sum_{j=1}^{N_r} j \cdot C_{\text{Ga}}^{i,j} C_{\text{As}}^{i,j},
\end{equation}
,где каждый элемент суммы соответствует участку площадью, пропорциональной $\Delta r^2 \cdot j$. Это стандартная аппроксимация интеграла в полярных координатах при радиальной симметрии.

После перехода к безразмерным концентрациям и введения параметра , уравнение принимает вид:

\begin{equation}
R_d^{i+1} = R_d^i - \frac{2 \Omega_{\text{Ga}} \omega C_0 \Delta r^2}{B(\theta) (R_d^i)^2} \sum_{j=1}^{N_r} j \cdot C_{\text{Ga}}^{i,j} C_{\text{As}}^{i,j}.
\end{equation}

Для упрощения записи удобно обозначить:

\begin{equation}
R_0^3 = \frac{2 \Omega_{\text{Ga}} \omega C_0 \Delta r^2}{B(\theta)},
\end{equation}

тогда итоговая формула будет:

\begin{equation}
R_d^{i+1} = R_d^i - \frac{R_0^3}{(R_d^i)^2} \sum_{j=1}^{N_r} j \cdot C_{\text{Ga}}^{i,j} C_{\text{As}}^{i,j}.
\end{equation}

\subsubsection{Рост кольца}

С учётом масштабирования по \( C_0 \), рост высоты кольца можно записать как:

\begin{equation}
\frac{d h(r,t)}{d t} = h_0 \cdot C_0 \cdot k_r \cdot C_{\text{Ga}}(r,t) \cdot C_{\text{As}}(r,t),
\end{equation}

Для численного расчёта это уравнение дискретизуется по времени с использованием явной схемы Эйлера:

\begin{equation}
h_{i+1,j} = h_{i,j} + \Delta t \cdot h_0 \cdot C_0 \cdot k_r \cdot C_{\text{Ga}}^{i,j} \cdot C_{\text{As}}^{i,j}.
\end{equation}

С учётом ранее введённого безразмерного параметра \( \omega = C_0 k_r \Delta t \), уравнение принимает окончательный вид:

\begin{equation}
h_{i+1,j} = h_{i,j} + \omega h_0 C_{\text{Ga}}^{i,j} C_{\text{As}}^{i,j}.
\end{equation}

Начальное условие для высоты кольца задаётся как:

\begin{equation}
h_{0,j} = 0.
\end{equation}

\subsubsection{Изменение радиуса капли}

Основываясь на объёме сферического сегмента и скорости расхода атомов, можно записать уравнение \eqref{eq:dRdt_integral} в виде:

\begin{equation}
R_d^{i+1} = R_d^i - \frac{2 \Omega_{\text{Ga}} k_r C_0^2 \Delta t \Delta r^2}{B(\theta) (R_d^i)^2} \sum_{j=1}^{N_r} j \cdot C_{\text{Ga}}^{i,j} C_{\text{As}}^{i,j},
\end{equation}
,где каждый элемент суммы соответствует участку площадью, пропорциональной $\Delta r^2 \cdot j$. Это стандартная аппроксимация интеграла в полярных координатах при радиальной симметрии.

После перехода к безразмерным концентрациям и введения параметра , уравнение принимает вид:

\begin{equation}
R_d^{i+1} = R_d^i - \frac{2 \Omega_{\text{Ga}} \omega C_0 \Delta r^2}{B(\theta) (R_d^i)^2} \sum_{j=1}^{N_r} j \cdot C_{\text{Ga}}^{i,j} C_{\text{As}}^{i,j}.
\end{equation}

Для упрощения записи удобно обозначить:

\begin{equation}
R_0^3 = \frac{2 \Omega_{\text{Ga}} \omega C_0 \Delta r^2}{B(\theta)},
\end{equation}

тогда итоговая формула будет:

\begin{equation}
R_d^{i+1} = R_d^i - \frac{R_0^3}{(R_d^i)^2} \sum_{j=1}^{N_r} j \cdot C_{\text{Ga}}^{i,j} C_{\text{As}}^{i,j}.
\end{equation}
    
\subsection{Программная реализация}

Математическая модель, описанная в предыдущих разделах, была реализована на языке \texttt{Rust}, что обеспечило высокую производительность, безопасную работу с памятью и удобную модульную структуру кода. Программа состоит из нескольких логически выделенных модулей:

\begin{itemize}
    \item \texttt{main.rs} — точка входа, инициализация параметров и запуск основного цикла расчёта;
    \item \texttt{geometry.rs} — вычисление потока галлия $F_{\text{Ga}}(r, t)$ и вспомогательные функции, связанные с геометрией капли;
    \item \texttt{solver.rs} — реализация схемы Эйлера для обновления концентраций $C_{\text{Ga}}$, $C_{\text{As}}$, высоты $h(r)$ и радиуса капли $R_d(t)$;
    \item \texttt{output.rs} — сохранение промежуточных и итоговых данных в текстовые файлы.
\end{itemize}

На каждом временном шаге расчёт выполняет следующий последовательный алгоритм:
\begin{enumerate}
    \item Расчёт профиля потока $F_{\text{Ga}}(r, t)$ вблизи границы капли;
    \item Обновление поверхностных концентраций $C_{\text{Ga}}(r)$ и $C_{\text{As}}(r)$;
    \item Вычисление прироста высоты кольца $h(r)$;
    \item Обновление радиуса капли $R_d(t)$ по формуле~(3.10);
    \item Сохранение текущих данных в выходные файлы.
\end{enumerate}

Функциональная реализация этих шагов отражена в следующем фрагменте кода \ref{listing-1}:

\begin{lstlisting}[language=Rust, caption={Основной цикл численного расчёта\label{listing-1}}]
for step in 0..nt {
    let flux = geometry::compute_flux(&state.r_d, &params);
    solver::update_concentrations(&mut state, &flux, &params);
    solver::update_height(&mut state, &params);
    solver::update_radius(&mut state, &params);
    output::save_step(&state, step);
}
\end{lstlisting}

Результаты моделирования сохраняются в текстовые файлы:
\begin{itemize}
    \item \texttt{droplet.dat} — зависимость радиуса капли $R_d(t)$;
    \item \texttt{height.dat} — профиль высоты кольца $h(r)$;
    \item \texttt{c\_ga.dat}, \texttt{c\_as.dat} — концентрации атомов Ga и As на поверхности.
\end{itemize}

Для последующего анализа и построения графиков используется отдельный скрипт на Python (\texttt{plot.py}), основанный на библиотеке \texttt{matplotlib}.

% ============================================
% ГЛАВА 4
% ============================================
\pagebreak
\section{Результаты и обсуждение}

\subsection{Анализ влияния температуры на формирование кольца}

Для анализа термодинамического поведения квантового кольца были проведены численные эксперименты при четырёх температурах: (а)~$T = 250^\circ$C, (б)~$T = 300^\circ$C, (в)~$T = 170^\circ$C и (г)~$T = 600^\circ$C. Во всех случаях варьировалась только температура, тогда как остальные параметры модели оставались неизменными (см. табл.~\ref{tab:params-fixed}). Ниже приведены графики временной эволюции концентраций атомов галлия и мышьяка, а также профиля высоты кольца.

\begin{table}
    \centering
    \caption{Постоянные параметры численного моделирования}
    \label{tab:params-fixed}
    \begin{tabular}{|l|l|l|}
    \hline
    \textbf{Параметр} & \textbf{Обозначение} & \textbf{Значение} \\ \hline
    Постоянная решётки & $a_0$ & 0.565\,нм \\ \hline
    Объём атома Ga & $\Omega_{\text{Ga}}$ &  $0{,}180\,\text{нм}^3$ \\ \hline
    Масштаб концентрации & $C_0$ & $ 3{,}13\,\text{нм}^{-2}$ \\ \hline
    Энергия активации Ga & $E_{\text{Ga}}$ & 1.0\,эВ \\ \hline
    Энергия активации As & $E_{\text{As}}$ & 0.5\,эВ \\ \hline
    Энергия десорбции As & $E_a$ & 1.0\,эВ \\ \hline
    Контактный угол капли & $\theta$ & $60^\circ$ \\ \hline
    Ширина потока Ga (гаусс) & $w$ & 3.0\,нм \\ \hline
    Начальный радиус капли & $R_d(0)$ & 30.0\,нм \\ \hline
    Коэффициент реакции & $k_r$ & 0.04\,нм$^2$/нс \\ \hline
    Относительный поток As & $F_{\text{As}}^{\text{rel}}$ & $10^{-10}$ \\ \hline
    \end{tabular}
\end{table}    

\begin{figure}
    \begin{center}
    \includegraphics[width=18cm]{images/C_Ga_t.png}
    \caption{\label{fig:c_ga_t} Распределение концентрации атомов галлия $C_{\text{Ga}}(r, t)$ при разных температурах: (а)~$T=250^\circ$C, (б)~$T=300^\circ$C, (в)~$T=170^\circ$C, (г)~$T=600^\circ$C.}
    \end{center}
\end{figure}

\begin{figure}
    \begin{center}
    \includegraphics[width=18cm]{images/C_As_t.png}
    \caption{\label{fig:c_as_t} Распределение концентрации атомов мышьяка $C_{\text{As}}(r, t)$ при разных температурах:(а)~$T=250^\circ$C, (б)~$T=300^\circ$C, (в)~$T=170^\circ$C, (г)~$T=600^\circ$C.}
    \end{center}
\end{figure}

\begin{figure}
    \begin{center}
    \includegraphics[width=18cm]{images/h-t.png}
    \caption{\label{fig:h_t} Профиль высоты кольца $h(r, t)$ во времени при различных температурах:(а)~$T=250^\circ$C, (б)~$T=300^\circ$C, (в)~$T=170^\circ$C, (г)~$T=600^\circ$C.}
    \end{center}
\end{figure}


\textbf{Низкая температура ($T = 170^\circ$C)} (рис.~\ref{fig:c_ga_t}в, \ref{fig:c_as_t}в, \ref{fig:h_t}в) соответствует кинетически ограниченному режиму, в котором рост структуры формируется преимущественно под действием локализованного потока галлия. Длина диффузии мышьяка в этом режиме составляет \textbf{196.9~нм}, что указывает на высокую подвижность атомов As, несмотря на низкую температуру.

График концентрации мышьяка $C_{\text{As}}(r, t)$ (рис.~\ref{fig:c_as_t}, в) показывает, что мышьяк активно накапливается на периферии, но его концентрация в центральной области существенно ниже — особенно на ранних этапах. Только к поздним временам профиль становится более пологим и приближается к насыщению. Это указывает на то, что несмотря на относительно высокую длину диффузии, распределение As по поверхности не является равномерным, а его доступность для реакции ограничена вблизи центра.

Однако профиль концентрации галлия $C_{\text{Ga}}(r, t)$ остаётся резко локализованным. Поток $F_{\text{Ga}}(r, t)$ сосредоточен вблизи границы капли и имеет узкую гауссову форму. При этом диффузия Ga ограничена. Следовательно, реакция Ga + As $\rightarrow$ GaAs происходит лишь в той зоне, где присутствуют оба компонента, то есть вблизи оси симметрии.

На графике высоты $h(r)$ это выражается как узкий, резкий пик в центре. Он быстро затухает с увеличением радиуса, а сама структура по форме напоминает компактную каплю с колоколообразным профилем. Такая морфология не является квантовой точкой строго, однако \textbf{по форме и механизму роста она стремится к точечной структуре}: высота сосредоточена в центре, периферия остаётся плоской.

Важно отметить, что основное ограничение здесь вносит не мышьяк, а именно галлий: его локальный поток и слабая диффузия определяют зону реакции. При прочих равных, доступность As по всей поверхности могла бы обеспечить рост более широкой структуры, однако из-за концентрации Ga только центральная область вовлечена в образование GaAs.

Сходные морфологии наблюдаются в экспериментальных работах, в частности, в~\cite{gurioli2021droplet}, где при $T \leq 180^\circ$C формируются симметричные куполообразные наноструктуры. Авторы отмечают, что недостаток подвижности Ga и ограниченность его поступления приводят к росту компактной точки даже при наличии As.

\textbf{Средняя температура ($T = 250^\circ$C)} (рис.~\ref{fig:c_ga_t}а, \ref{fig:c_as_t}а, \ref{fig:h_t}а) соответствует переходному режиму, в котором создаются благоприятные условия для устойчивого роста одиночного квантового кольца. Здесь достигается оптимальный баланс между поступлением компонентов, их поверхностной диффузией и скоростью реакции.

Численно определённая длина диффузии мышьяка составляет \textbf{72{,}33~нм}, что обеспечивает достаточно эффективное распространение по поверхности подложки. График $C_{\text{As}}(r, t)$ (рис.~\ref{fig:c_as_t}, а) показывает, что на ранних стадиях роста концентрация мышьяка в центре остаётся пониженной, однако по мере времени профиль выравнивается, становясь пологим и близким к насыщению. Таким образом, к поздним моментам времени мышьяк охватывает всю активную область, включая центр и периферию, что создаёт условия для пространственно широкой реакции с галлием.

В отличие от мышьяка, галлий демонстрирует совершенно иную картину. Профиль $C_{\text{Ga}}(r, t)$ (рис.~\ref{fig:c_ga_t}, а) указывает на чёткую локализацию вещества вблизи центра ($r = 0$): концентрация резко падает и практически исчезает уже к $r \approx 20$–25\,нм. Это свидетельствует о том, что поток $F_{\text{Ga}}(r, t)$ сохраняет узкий гауссов профиль, сосредоточенный в центральной зоне, и почти не распространяется по поверхности. Диффузия галлия остаётся ограниченной даже при данной температуре, и именно это ограничивает зону, в которой возможно образование твёрдой фазы GaAs.

Реакция между компонентами возможна только в области их перекрытия, которая формируется на краю зоны галлия — на расстоянии порядка 15–25~нм от центра. Это приводит к пространственно-избирательному росту, при котором центр остаётся незаполненным, а максимальный рост наблюдается по периферии.

На графике высоты $h(r)$ (рис.~\ref{fig:h_t}а) это проявляется в виде характерной кольцевой формы с чётким центральным минимумом и симметричным максимумом на расстоянии порядка 40–50~нм. Профиль кольца гладкий, устойчивый, без вторичных пиков, что говорит о стабильной кинетике роста. Кольцевая морфология при этом формируется не из-за растяжения потока Ga, а исключительно благодаря высокой подвижности мышьяка.

Сравнивая с режимом при $T = 170^\circ$C, где галлий и мышьяк локализованы в центре, здесь наблюдается принципиально иная картина: рост смещён в радиальном направлении, образуя кольцевую структуру. Такая морфология согласуется с экспериментальными результатами, в частности, с работой~\cite{mano2005self}, где аналогичные GaAs-кольца были получены при умеренных температурах методом капельной эпитаксии.

Таким образом, температурный режим при $T = 250^\circ$C можно считать \textbf{оптимальным для формирования одиночного симметричного квантового кольца} с чётким центральным отверстием, устойчивым профилем и хорошо контролируемыми размерами.

\textbf{Переходная температура ($T = 300^\circ$C)} (рис.~\ref{fig:c_ga_t}б, \ref{fig:c_as_t}б, \ref{fig:h_t}б) представляет собой критический режим, при котором морфология кольца перестраивается. Он характеризуется смещением реакции на периферию, неравномерной кристаллизацией и признаками зарождающейся двойной кольцевой структуры.

Численно рассчитанная длина диффузии мышьяка составляет \textbf{44{,}58~нм}, что меньше, чем при $T = 250^\circ$C. Это указывает на ограниченную подвижность As: на графике $C_{\text{As}}(r, t)$ видно, что на ранних этапах концентрация мышьяка близка к нулю в центральной зоне и начинает возрастать только начиная с $r \approx 70$–80\,нм. По мере времени фронт продвигается внутрь, но даже на финальных срезах насыщение вблизи центра остаётся неполным. Таким образом, максимальная реакционная активность смещена на периферию, где происходит перекрытие с потоком галлия.

В то же время поведение галлия становится заметно отличным. Профиль $C_{\text{Ga}}(r, t)$ демонстрирует широкое распределение и наличие периферийного максимума — концентрация достигает пика не в центре, а на расстоянии 30–40~нм. Это связано с изменением геометрии капли по мере роста: радиус \( R_d(t) \) уменьшается, а поток $F_{\text{Ga}}(r, t)$ перераспределяется и расширяется. Таким образом, зона, где Ga и As одновременно присутствуют, смещается от центра.

На графике высоты кольца $h(r)$ (рис.~\ref{fig:h_t}б) это отражается в виде широкого, пологого профиля с несимметричной формой. На поздних стадиях наблюдается намечающийся вторичный максимум, а центральный провал становится менее выраженным. Хотя структура ещё не демонстрирует классического двойного кольца, форма явно стремится к нему — появляются две пространственно разделённые зоны роста: внутренний и внешний обод.

Такая морфология может быть отнесена к переходному типу: между одиночным кольцом и двойным кольцом. В литературе, например, в работе~\cite{mano2005self}, описано образование концентрических GaAs-колец при умеренно высоких температурах. Ключевым фактором при этом является пространственное разделение потоков и конкуренция двух зон кристаллизации — именно это и наблюдается в численном моделировании при $T = 300^\circ$C.

\textbf{Высокая температура ($T = 600^\circ$C)} (рис.~\ref{fig:c_ga_t}г, \ref{fig:c_as_t}г, \ref{fig:h_t}г) соответствует режиму, при котором рост квантовой структуры существенно подавлен вследствие интенсивной термической десорбции мышьяка. Вопреки распространённым ожиданиям, повышение температуры не приводит к увеличению эффективности роста, а напротив — резко ограничивает его.

Численно рассчитанная длина диффузии мышьяка составляет всего \textbf{7.83~нм} — минимальное значение среди всех рассмотренных температурных режимов. Это связано с крайне малым временем жизни атомов As на поверхности: они не успевают вступить в реакцию и быстро десорбируются. График $C_{\text{As}}(r, t)$ подтверждает это: концентрация мышьяка практически отсутствует в центральной области до $r \approx 100$~нм и начинает расти лишь на периферии, что свидетельствует о невозможности эффективной диффузии к центру и смещении фронта осаждения далеко от капли.

Профиль галлия $C_{\text{Ga}}(r, t)$ (рис.~\ref{fig:c_ga_t}г) сохраняет ожидаемую форму: концентрация максимальна вблизи центра и постепенно убывает на больших расстояниях. Однако в отсутствие достаточного количества мышьяка реакция Ga + As $\rightarrow$ GaAs практически не происходит. Галлий присутствует на поверхности, но не участвует в кристаллизации.

Это отражается в графике высоты $h(r)$ (рис.~\ref{fig:h_t}г): значения остаются на уровне $h = 0{,}002$, рост крайне слабый, профиль пологий и размазанный по радиусу. Высота не демонстрирует выраженного пика или кольцевой формы, отсутствует центральный провал — вместо этого наблюдается монотонное расширение с очень низкой амплитудой.

Тем не менее, при частичной компенсации десорбции (например, увеличением флюкса мышьяка, понижением давления в камере или использованием импульсной подачи) возможен ограниченный рост. В этом случае структура формируется с равномерной малой высотой, без чёткой радиальной избирательности. Такая морфология приближается к \textbf{квантовому диску} — сплошной, слабовыраженной наноструктуре без центрального отверстия и с равномерным распределением осаждённого материала. Переход от кольцевой к дисковой морфологии при высоких температурах также описан в работе~\cite{fan2023evaporation}, где отмечается потеря топологического отверстия и полное сглаживание профиля кольца.

Таким образом, длина диффузии мышьяка и характер распределения потоков являются ключевыми параметрами, определяющими эффективность и тип роста при капельной эпитаксии. При низких ($T = 170^\circ$C) и высоких ($T = 600^\circ$C) температурах рост ограничен: в первом случае из-за локализации галлия, во втором — вследствие быстрой десорбции мышьяка. Только в интервале температур $T = 250$–$300^\circ$C достигается баланс между диффузией, реакцией и поступлением вещества, при котором формируется устойчивая кольцевая морфология. В этом диапазоне наблюдаются как симметричные одиночные кольца, так и переход к более сложным структурам, включая двойные кольца.

\subsection{Анализ влияния потока мышьяка на формирование кольца}

Для оценки влияния интенсивности потока атомов мышьяка на морфологию квантового кольца была проведена серия численных экспериментов с фиксированной температурой \(T = 250^\circ C\), геометрией капли и другими параметрами модели. Варьировался лишь параметр флюкса \( F_{\text{As}} \), с тремя значениями: (а)~\(3 \times 10^{-12}\) нм$^{-2}$с$^{-1}$, (б)~\(1 \times 10^{-10}\) нм$^{-2}$с$^{-1}$ (базовый режим), (в)~\(4 \times 10^{-10}\) нм$^{-2}$с$^{-1}$.

Результаты моделирования представлены на рис.~\ref{fig:ga_flux_2}–\ref{fig:h_flux_2} и демонстрируют распределения концентраций галлия, мышьяка, а также профили высоты кольца на различных временных срезах.

\begin{figure}
    \begin{center}
    \includegraphics[width=18cm]{images/C_Ga_t_2.png}
    \caption{\label{fig:ga_flux_2} Распределение концентрации галлия $C_{\text{Ga}}(r)$ во времени при различных потоках мышьяка: (а)~$F_{\text{As}} = 2 \times 10^{-11}$ нм$^{-2}$с$^{-1}$, (б)~$1 \times 10^{-10}$ нм$^{-2}$с$^{-1}$, (в)~$4 \times 10^{-10}$ нм$^{-2}$с$^{-1}$.}
    \end{center}
\end{figure}

\begin{figure}
    \begin{center}
    \includegraphics[width=18cm]{images/C_As_t_2.png}
    \caption{\label{fig:as_flux_2} Распределение концентрации мышьяка $C_{\text{As}}(r)$ во времени при различных потоках $F_{\text{As}}$: (а)~$2 \times 10^{-11}$, (б)~$1 \times 10^{-10}$, (в)~$4 \times 10^{-10}$ нм$^{-2}$с$^{-1}$.}
    \end{center}
\end{figure}

\begin{figure}
    \begin{center}
    \includegraphics[width=18cm]{images/h_t_2.png}
    \caption{\label{fig:h_flux_2} Профиль высоты кольца $h(r)$ во времени при различных потоках мышьяка: (а)~$F_{\text{As}} = 2 \times 10^{-11}$, (б)~$1 \times 10^{-10}$, (в)~$4 \times 10^{-10}$ нм$^{-2}$с$^{-1}$.}
    \end{center}
\end{figure}


\paragraph{Сильно пониженный поток мышьяка (\(3 \times 10^{-12}\) нм$^{-2}$с$^{-1}$).}

При экстремально низком потоке мышьяка поступление As на поверхность практически прекращается, что приводит к резкому замедлению реакции Ga + As $\rightarrow$ GaAs. Это видно из распределения концентрации мышьяка $C_{\text{As}}(r)$ (рис.~\ref{fig:as_flux_2},~а), где даже к поздним временам профиль остаётся крайне низким и достигает насыщения только на очень больших расстояниях ($r > 150$ нм). Центральная область остаётся практически пустой по As, что исключает реакцию в зоне действия потока галлия.

На графике $C_{\text{Ga}}(r)$ (рис.~\ref{fig:ga_flux_2},~а) видно, что галлий сохраняет характерную гауссову форму и не испытывает значительных изменений — поток из капли работает, но без мышьяка он не может привести к росту кристаллической фазы.

Профиль высоты $h(r)$ (рис.~\ref{fig:h_flux_2},~а) иллюстрирует критическое снижение роста: структура формируется только в виде едва различимого приподнятого ободка с высотой порядка $0.009$ нм. При этом морфология остаётся однотонной, асимметричной, с постепенным спадом, и не проявляет признаков кольцевой структуры.

Это подтверждает, что при столь малом значении $F_{\text{As}}$ наблюдается не просто торможение роста, а его почти полная остановка, несмотря на наличие галлия. Основная масса Ga остаётся недореагировавшей. Морфология в этом режиме может напоминать застывшее начальное состояние, в котором кристаллизация не успела начаться до десорбции As.

Важно отметить, что такая ситуация противоположна, например, температурному режиму $T = 300^\circ$C, при котором рост смещается на периферию, но сохраняется активным. Здесь же — отсутствует активная зона реакции вовсе. Это подтверждает критическую роль флюкса мышьяка в инициировании кристаллизации при капельной эпитаксии.

В литературе аналогичные режимы исследованы в работе~\cite{schmidt2021}, где показано, что при значениях флюкса ниже $10^{-11}$ нм$^{-2}$с$^{-1}$ рост GaAs практически не происходит даже при наличии жидкой капли. Морфология в этом случае определяется не диффузией, а ограничением поступления As.

\paragraph{Базовый поток мышьяка (\(1 \times 10^{-10}\) нм$^{-2}$с$^{-1}$).}

Данный режим соответствует оптимальному значению флюкса мышьяка, при котором достигается сбалансированное соотношение между поступлением As и Ga, а также эффективная реакция образования GaAs. В этом случае наблюдается формирование устойчивого и симметричного квантового кольца с выраженным центральным отверстием.

Подробный анализ данного случая в настоящем разделе не приводится, поскольку он полностью совпадает с режимом температурного моделирования при $T = 250^\circ$C, рассмотренным ранее в разделе 4.1. Графики $C_{\text{As}}(r)$, $C_{\text{Ga}}(r)$ и $h(r)$, соответствующие данному потоку, уже представлены в рамках анализа температурного влияния (см. рис.~\ref{fig:c_ga_t}, \ref{fig:c_as_t}, \ref{fig:h_t}).

Таким образом, здесь он используется в качестве базового режима для сравнения с другими значениями потока As. Морфология, полученная в этом случае, согласуется с результатами экспериментальных исследований~\cite{han2019,lee2022}, в которых подобные кольца наблюдаются при оптимальном V/III соотношении в условиях капельной эпитаксии GaAs.

\paragraph{Повышенный поток мышьяка (\(4 \times 10^{-10}\) нм$^{-2}$с$^{-1}$).}

При высоком значении флюкса As насыщение поверхности мышьяком достигается уже на ранних временных этапах, что видно на графике $C_{\text{As}}(r)$ (рис.~\ref{fig:as_flux_2},~в). Однако из-за ограниченной подвижности атомов галлия (ограниченной длины его диффузии) активная зона реакции смещается ближе к центру, где локализован поток Ga.

Это приводит к изменению морфологии: на графике высоты $h(r)$ (рис.~\ref{fig:h_flux_2},~в) видно формирование компактной куполообразной структуры без выраженного центрального провала. В отличие от кольца, характерного для сбалансированного режима, здесь наблюдается монотонный максимум в центре, что свидетельствует о тенденции к образованию квантовой точки.

Похожее поведение описано в работе~\cite{okada2020}, где показано, что при избытке мышьяка рост GaAs смещается в область над каплей, приводя к появлению точечной морфологии. Такая трансформация связана с тем, что высокое насыщение As приводит к немедленной реакции на месте поступления Ga, подавляя боковую диффузию и исключая возможность кольцеобразного роста.

Интересно, что аналогичная морфология наблюдалась ранее при пониженной температуре (\(T = 170^\circ\)C), рассмотренной в разделе 4.1. Там ограничивающим фактором была не избыточная реакция, а слабая диффузия Ga при низкой температуре. В обоих случаях (высокий $F_{\text{As}}$ и низкое $T$) формируется точечноподобная структура за счёт ограничения пространственного распределения компонентов: либо Ga не успевает распространиться, либо As немедленно улавливается при контакте.

Таким образом, в режиме повышенного потока As морфология вновь отклоняется от кольцевой и стремится к типичной геометрии квантовой точки, подтверждая, что баланс диффузии и поступления компонентов критически важен для формирования нужной структуры.

\subsection{Анализ влияния начального радиуса капли на формирование кольца}

Одним из ключевых параметров, определяющих морфологию формируемой структуры при капельной эпитаксии, является начальный радиус капли \( R_d(0) \). Он определяет ширину и интенсивность потока галлия \( F_{\text{Ga}}(r, t) \), объём доступного материала, а также геометрию зоны, в которой происходит реакция с атомами мышьяка. В данной серии моделирования рассмотрены три значения начального радиуса капли: (а)~15 нм, (б)~30 нм (базовое), (в)~60 нм. Результаты представлены на рис.~\ref{fig:rd_c_as},~\ref{fig:rd_c_ga},~\ref{fig:rd_h}.

\begin{figure}
    \begin{center}
    \includegraphics[width=18cm]{images/C_As_t_3.png}
    \caption{\label{fig:rd_c_as} Распределение концентрации мышьяка $C_{\text{As}}(r)$ при разных значениях начального радиуса капли: (а)~15 нм, (б)~30 нм, (в)~60 нм.}
    \end{center}
\end{figure}

\begin{figure}
    \begin{center}
    \includegraphics[width=18cm]{images/C_Ga_t_3.png}
    \caption{\label{fig:rd_c_ga} Распределение концентрации галлия $C_{\text{Ga}}(r)$ при разных $R_d(0)$.}
    \end{center}
\end{figure}

\begin{figure}
    \begin{center}
    \includegraphics[width=18cm]{images/h_t_3.png}
    \caption{\label{fig:rd_h} Профиль высоты кольца $h(r)$ во времени при разных начальных радиусах капли.}
    \end{center}
\end{figure}

\paragraph{(а) Малый радиус капли \(R_d = 15\) нм.}
Малый размер капли приводит к резкому локальному потоку Ga, сконцентрированному в центральной зоне. Это видно по профилю концентрации \(C_{\text{Ga}}(r)\) (рис.~\ref{fig:rd_c_ga},~а), где максимум располагается вблизи \(r = 10\) нм и быстро спадает. Концентрация мышьяка (рис.~\ref{fig:rd_c_as},~а) относительно равномерна, что способствует локализованной реакции в центре. Результатом является формирование высокой и узкой структуры (рис.~\ref{fig:rd_h},~а), по форме напоминающей квантовую точку или узкое кольцо. Подобная морфология описана в работе~\cite{kuroda2018tunable}, где уменьшение объёма капли Ga при капельной эпитаксии GaAs приводило к точечной геометрии с высоко локализованным пиком.

\paragraph{(б) Базовый радиус капли \(R_d = 30\) нм.}
Этот режим соответствует сбалансированным условиям, ранее рассмотренным в разделе 4.1. Поток Ga имеет среднюю ширину, а зона реакции охватывает область \(r \approx 25\text{–}50\) нм. Концентрации компонентов пересекаются в области максимальной реакции, формируя симметричное кольцо с выраженным центральным провалом (рис.~\ref{fig:rd_h},~б). Эта морфология типична для одиночных квантовых колец и подтверждается как экспериментами, так и моделированием~\cite{zhang2020morphology}.

\paragraph{(в) Увеличенный радиус капли \(R_d = 60\) нм.}
Увеличение радиуса капли ведёт к растяжению гауссового профиля потока галлия и смещению зоны реакции на периферию (рис.~\ref{fig:rd_c_ga},~в). Концентрация мышьяка остаётся достаточно равномерной, и кристаллизация происходит в более широкой области, приводя к формированию размытого, блюдцеобразного профиля (рис.~\ref{fig:rd_h},~в). Подобные широкие структуры с приглушенным кольцевым контуром наблюдаются в экспериментах~\cite{wang2021droplet}, где увеличение объёма капли приводило к снижению локальной плотности потока Ga и формированию низкопрофильных колец.

%  ВЫВОДЫ И ЗАКЛЮЧЕНИЕ
% ============================================
\pagebreak
\specialsection{Выводы}
Структура файлов, которые можно редактировать:

\begin{itemize}
    \item \verb|diploma.tex| --- содержит основной текст;
    \item \verb|titlepage.tex| --- содержит титульный лист;
    \item \verb|literature.bib| --- содержит источники для списка литературы;
    \item \verb|code_highlight.tex| --- форматирование листингов (фрагментов кода).
\end{itemize}

Файл \verb|style.tex| очень важный, его трогать и особенно удалять не надо, там задаются различные стили документа. Редактировать в случае, если знаете, что делать.

\specialsection{Заключение}

Нужны ли отдельно и выводы, и заключение --- я не знаю. Разберёмся.

Список литературы ниже оформлен не совсем по ГОСТу, но это легко исправить. Главное, что он организован, и можно ссылаться на каждый пункт по фамилии первого автора.

\textbf{Внимание!} 

Список литературы находится в отдельном файле \verb|literature.bib|, в который можно добавлять новые источники в любом порядке. Они будут сами располагаться как нужно, в порядке упоминания в тексте.

Если какой-то источник не процитирован в тексте, он в список литературы добавлен не будет.

Поэтому один и тот же файл с источниками можно использовать для нескольких документов.


\pagebreak
\printbibliography

\end{document}
